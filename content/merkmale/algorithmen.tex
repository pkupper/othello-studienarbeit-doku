%!TEX root = ../../dokumentation.tex

\section{Verbesserungen durch Pruning}

Um den Einfluss von Alpha-Beta-Pruning, Iterative Deepening und ProbCut auf die Leistung der \ac{KI} beurteilen zu
können, kann die Zeit gemessen werden, die bei einem gegebenen Zustand benötigt wird, um den nächsten Zug zu berechnen.
Diese wurde für verschiedene Suchtiefen berechnet und in Tabelle \ref{table:calctimes} zusammengefasst. Hier ist
deutlich zu erkennen, dass der Alpha-Beta-Algorithmus durch das Anwenden von Pruning im Gegensatz zum reinen
Minimax-Algorithmus vor allem für größere Suchtiefen deutlich weniger Zeit benötigt. Alpha-Beta mit Iterative Deepening
hat trotz Overhead durch die bessere Sortierung sehr ähnliche Laufzeiten wie Alpha-Beta. Für ProbCut wurde eine flache
Suchtiefe von 2 und eine tiefe Suchetiefe von 4 verwendet, sodass Effekte erst ab einer Suchtiefe von 6 auftreten. Die
Performance-Einsparungen von ProbCut sind allerdings geringer als der zusätzliche Aufwand durch die flachen Suchen,
sodass besonders mit Iterative Deepening mehr Zeit für die Berechnung des nächsten Zuges benötigt wird, als mit dem
Alpha-Beta Algorithmus.

\begin{table}[hb]
\centering
\begin{tabular}{c|ccccc}
\hline
\diagbox{Tiefe}{KI} & Minimax & Alpha-Beta & AB\,+\,ID & ProbCut & PC\,+\,ID \\ \hline
3 & 0.35 & 0.35 & 0.36 & 0.34 & 0.37 \\
4 & 2.71 & 1.28 & 1.21 & 1.39 & 1.26 \\
5 & 22.76 & 5.66 & 5.97 & 5.63 & 5.95 \\
6 & 194 & 18.1 & 18.2 & 18.9 & 19.4 \\
7 & 1786 & 72.9 & 71.3 & 74.3 & 81.0 \\
8 & - & 242 & 221 & 252 & 270 \\
\end{tabular}
\caption{Benötigte Zeit in Sekunden für die Berechnung des nächsten Zuges}
\label{table:calctimes}
\end{table}

Tabelle \ref{table:numstates} zeigt, wie oft die Heuristik beim Berechnen des nächsten Zuges mit unterschiedlichen
Suchtiefen ausgewertet wurde. Hier ist erkennbar, dass bei Verwendung von Iterative Deepening die Heuristik seltener
auswerten muss. Das liegt daran, dass durch die bessere Sortierung gute Züge früher betrachtet werden und mehr Zweige
ausgeschlossen werden können. Beim Vergleich von Alpha-Beta und ProbCut bei Suchtiefe 7 und 8 sieht man, dass auch hier
teilweise Zweige nicht betrachtet werden und die Heuristik seltener ausgewertet werden muss. Es fällt allerdings auch
auf, dass ProbCut vor allem in Kombination mit Iterative Deepening keinen Vorteil erzielen kann.

\begin{table}[hb]
\centering
\begin{tabular}{c|ccccc}
\hline
\diagbox{Tiefe}{KI} & Minimax & Alpha-Beta & AB\,+\,ID & ProbCut & PC\,+\,ID \\ \hline
3 & 461 & 490 & 499 & 490 & 499  \\
4 & 3487 & 1720 & 1523 & 1720 & 1523 \\
5 & 28806 & 7387 & 7344 & 7387 & 7344 \\
6 & 243815 & 23432 & 21931 & 24158 & 23334 \\
7 & 2193633 & 92059 & 85934 & 91514 & 96345 \\
8 & - & 301806 & 262391 & 301299 & 310938 \\
\end{tabular}
\caption{Anzahl Auswertungen der Heuristik}
\label{table:numstates}
\end{table}
