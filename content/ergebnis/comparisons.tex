\section{Stärke der KI}
Dieses Kapitel beschäftigt sich mit dem Vergleich von Parametern und Strategien, die sich auf die Stärke der KI auswirken.

\subsection{Suchtiefe}
Ein wichtiger Faktor für die Stärke der KI ist die erreichte Suchtiefe.
Die Erhöhung der Suchtiefe um eins führt dazu, dass die KI, je nach Gesamtsuchtiefe, zu 80~–~95\,\% gegen die KI mit geringerer Suchtiefe gewinnt.
Diese Zahlen wurden ermittelt, indem die gleiche KI jeweils 20 Mal mit unterschiedlicher Suchtiefe gegeneinander spielen gelassen wurde.
Die Ergebnisse befinden sich in der Tabelle \ref{table:search-depth}.
Es ist also wichtig, eine möglichst hohe Suchtiefe in akzeptabler Zeit zu erreichen.
Bei begrenzter Rechenzeit kann eine höhere Performanz zu einer stärkeren KI führen.
%TODO: Tabelle in Anhang?

\begin{table}[hb]
    \centering
    \begin{tabular}{c|ccc}
    \hline
    Suchtiefe Schwarz\,/\,Weiß & Siege Schwarz & Unentschieden & Siege Weiß \\ \hline
    1\,/\,2  & 15\,\% &  0\,\% & 85\,\% \\
    2\,/\,3  &  5\,\% &  0\,\% & 95\,\% \\
    3\,/\,4  & 10\,\% &  0\,\% & 90\,\% \\
    4\,/\,5  & 15\,\% &  5\,\% & 80\,\% \\
    \hline
    \end{tabular}
    \caption{Anteil gewonnener Spiele bei unterschiedlicher Suchtiefe}
    \label{table:search-depth}
\end{table}
