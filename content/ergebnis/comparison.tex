%!TEX root = ../../dokumentation.tex

\section{Vergleich mit anderen KIs}
Dieses Kapitel beschäftigt sich mit dem Vergleich mit anderen Othello KIs.

\subsection{Vorgänger}
Im letzten Jahr wurde bereits eine Studienarbeit von anderen Studenten über die Entwicklung einer Othello-KI verfasst.
In dieser wurden im Wesentlichen zwei alternative KIs implementiert.
\begin{enumerate}
    \item Die erste KI nutzt den Monte-Carlo Algorithmus. Um den nächsten Spielzug zu berechnen, wird für alle möglichen
    Züge das Spiel $n$ Mal zufällig bis zum Ende gespielt.
    \cite[S.~19]{othellustudienarbeit}
    \item Die zweite KI basiert auf dem AlphaBeta-Algorithmus und kann verschiedene Heuristiken verwenden. Am
    erfolgreichsten war die „Stored Monte-Carlo-Heuristik“, die auf statistischen Daten über die
    Siegeswahrscheinlichkeit beim Setzen auf ein bestimmtes Feld in einem bestimmten Zug beruht.
    \cite[S.~30]{othellustudienarbeit}
\end{enumerate}
Die Arbeit ist zu dem Ergebnis gekommen, dass der Monte-Carlo-Algorithmus AlphaBeta überlegen ist.
\cite[S.~55]{othellustudienarbeit}

Tabelle \ref{table:comp:previous} zeigt einen Vergleich der in dieser Arbeit entwickelten KI mit den oben genannten Strategien. Dabei wurde sowohl für ProbCut als auch für AlphaBeta eine Suchtiefe von 5 verwendet.

\begin{table}[hb]
\centering
\begin{tabular}{c|c|ccc}
\hline
Gegnerische KI & \#Spiele & Siege ProbCut & Unentschieden & Siege Gegner \\
\hline
 Monte-Carlo & 20 & \,\% &  \,\% & \,\% \\ %TODO: Werte
 AlphaBeta   & 20 & 100\,\% &   0\,\% &  0\,\% \\
\hline
\end{tabular}
\caption{Anteil der Siege und Niederlagen von ProbCut gegen die KI der Vorgängerarbeit}
\label{table:comp:previous}
\end{table}
