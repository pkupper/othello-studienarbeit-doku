%!TEX root = ../../dokumentation.tex

\section{Vorgänger}
Im letzten Jahr wurde bereits von zwei Studierenden eine Studienarbeit über die Entwicklung einer Othello-\ac{KI} verfasst.
In dieser wurden im Wesentlichen zwei alternative \ac{KI}-Strategien implementiert.
\begin{enumerate}
    \item Die erste \ac{KI} nutzt den Monte-Carlo Algorithmus. Um den nächsten Spielzug zu berechnen, wird für alle möglichen
    Züge das Spiel $n$ Mal zufällig bis zum Ende gespielt.
    \cite[S.~19]{othellostudienarbeit}
    \item Die zweite \ac{KI} basiert auf dem Alpha-Beta-Algorithmus und kann verschiedene Heuristiken verwenden. Am
    erfolgreichsten war die „Stored Monte-Carlo-Heuristik“, die auf statistischen Daten über die
    Siegeswahrscheinlichkeit beim Setzen auf ein bestimmtes Feld in einem bestimmten Zug beruht.
    \cite[S.~30]{othellostudienarbeit}
\end{enumerate}
Die Arbeit ist zu dem Ergebnis gekommen, dass die Implementierung des Monte-Carlo-Algorithmus der des
Alpha-Beta-Algorithmus überlegen ist.
\cite[S.~55]{othellostudienarbeit}

Tabelle \ref{table:comp:previous} zeigt einen Vergleich der in dieser Arbeit entwickelten \ac{KI} mit den oben genannten
Strategien. Dabei wurde sowohl für ProbCut als auch für Alpha-Beta eine Suchtiefe von 5 verwendet. Mit 100\,\% Siegen
ist die in dieser Arbeit entwickelte \ac{KI} deutlich überlegen.

\begin{table}[H]
\centering
\begin{tabular}{c|c|ccc}
\hline
Gegnerische \ac{KI} & \#Spiele & Siege ProbCut & Unentschieden & Siege Gegner \\
\hline
Monte-Carlo & 20 & 100\,\% & 0\,\% & 0\,\% \\
Alpha-Beta  & 20 & 100\,\% & 0\,\% & 0\,\% \\
\hline
\end{tabular}
\caption{Anteil der Siege und Niederlagen von ProbCut gegen die \ac{KI} der Vorgängerarbeit}
\label{table:comp:previous}
\end{table}

\section{Cowthello}

Die Cowthello \ac{KI} nutzt im Wesentlichen die Cowthello Heuristik, die auch in der in dieser Arbeit entwickelten \ac{KI}
verwendet wird, passt diese jedoch an, wenn Ecken des Spielfelds belegt werden. \cite{cowthello}

Im Spiel gegen die in dieser Arbeit entwickelte ProbCut Strategie mit einer festen Suchtiefe von 5 Halbzügen gewinnt
die ProbCut Strategie, wie in Tabelle \ref{table:comp:cowthello} dargestellt, in 100\,\%  der Spiele.

\begin{table}[H]
\centering
\begin{tabular}{c|c|ccc}
\hline
Gegnerische \ac{KI} & \#Spiele & Siege ProbCut & Unentschieden & Siege Gegner \\
\hline
Cowthello & 5 & 100\,\% & 0\,\% & 0\,\% \\
\hline
\end{tabular}
\caption{Anteil der Siege und Niederlagen von ProbCut gegen die Cowthello \ac{KI}}
\label{table:comp:cowthello}
\end{table}