\section{Der Minimax-Algorithmus}

Minimax ist ein Algorithmus zur Bestimmung des optimalen Spielzugs in einem deterministischen, rundenbasierten, zwei-Spieler Nullsummenspiel mit perfekter Information.

Dabei wird für jeden Knoten im Spielbaum des Spiels, durch Rekursion, der Nutzen des entsprechenden Spielzustands für alle Spieler bestimmt.
An den Blatt-Knoten des Spielbaums ist der Nutzen für alle Spieler, in Form des Spielergebnisses, bereits bekannt.
Die Nutzen-Werte der restlichen Knoten werden bestimmt, indem, unter der Annahme das beide Spieler optimal spielen, immer der größte Nutzen aus den Kind-Knoten für den aktuell spielenden Spieler ausgewählt wird.
\cite[S.~165]{ai2010russel}

Dadurch lässt sich zu jedem Spielzustand der Zug bestimmen, der gegen einen optimal spielenden Gegner von größtem Nutzen ist.

Da bei einem zwei-Spieler Nullsummenspiel die Summe der Nutzen der beiden Spieler konstant ist, und sich somit der Nutzen für einem Spieler aus dem Nutzen des anderen Spielers ergibt, muss nur der Nutzen für einen Spieler betrachtet werden. Dieser wird dann abwechselnd von diesem Spieler maximiert, und von dessen Gegner minimiert.
Die Spieler werden dann maximierender und einen minimierender Spieler genannt.
Aus diesem Vorgehensweise ergibt sich der Name "'Minimax"' für die zwei-Spieler Variante des Algorithmus.
Bei Spielen mit mehr als zwei Spielern kann diese Optimierung nicht angewandt werden, weshalb in diesem Fall immer der Nutzen für alle Spieler getrennt betrachtet werden muss.
\cite[S.~165]{ai2010russel}

%TODO: Mathematische Beschreibung / evtl. Implementation



