\section{Heuristische Evaluationsfunktion}
\label{sec:heuristic}
Dieser Abschnitt behandelt die heuristische Evaluationsfunktion, die zur Approximation des Nutzens eines Spielzustands
verwendet wird.

Optimalerweise würde die Evaluationsfunktion den tatsächlichen Nutzen einer Spielsituation bestimmen. Eine solche
Evaluationsfunktion ist zwar durch den Minimax-Algorithmus gegeben, dieser kann jedoch aus offensichtlichen Gründen
nicht zum Einsatz kommen, da der Rechenaufwand hierbei identisch zu einer unbeschränkten Suche wäre.

Um den Nutzen verschiedener Spielzüge miteinander zu vergleichen, wird also eine weniger rechenaufwändige
Evaluationsfunktion benötigt, die den Nutzen nur ausreichend gut annähert, statt den tatsächlichen Nutzen zu bestimmen.

\subsection{Anzahl der Steine}
\label{sec:disccount}
Ein primitiver Ansatz für eine Evaluationsfunktion bei dem Spiel Othello ist die Differenz der Anzahl von eigenen
Steinen und gegnerischen Steinen auf dem Spielfeld, da diese am Ende über den Sieg entscheidet. Gegen Ende eines Spiels
könnte dieser Ansatz eine passende Bewertung liefern, allerdings kann es vor allem in der Anfangsphase sinnvoll sein,
weniger Steine zu haben. Viel wichtiger ist z.\,B. der Besitz von Ecken oder die Möglichkeit, neue Steine platzieren zu
können.

\subsection{Mobilität}
\label{sec:mobility}
Ziel der Mobilität ist es, dass der Spieler möglichst viele Steine platzieren kann. Dadurch kann dessen
Entscheidungsfreiheit maximiert werden, während die Freiheit des Gegners durch eine geringe Anzahl an möglichen Zügen
eingeschränkt wird. Es wird unterschieden zwischen aktueller und potenzieller Mobilität, auf die im Folgenden
eingegangen wird.
\cite[S.~7f.]{evaluationfunctions}

\subsubsection{Aktuelle Mobilität}
\label{sec:theorycurrentmobility}
Die aktuelle Mobilität ist ein Maß dafür, wie viele mögliche Züge die Spieler gerade haben. Eine niedrige Anzahl an
möglichen Zügen ist häufig ungünstig, da der Spieler dann möglicherweise gezwungen ist, einen weniger guten Zug zu
machen. Als quantitatives Merkmal kann zunächst die Differenz der Anzahl möglicher Züge beider Spieler betrachtet
werden. Bei gleicher Differenz verfügt in der Regel der Spieler über die bessere Stellung, der mehr mögliche Züge hat
als sein Gegner. Um diese Tatsache zu berücksichtigen, kann eine nicht-lineare Funktion verwendet werden.
\cite[S.~7]{evaluationfunctions}

\subsubsection{Potenzielle Mobilität}
\label{sec:potmobility}
Neben der aktuellen Mobilität kann auch die potenzielle Mobilität betrachtet werden. Diese ist ein Indikator für die
mögliche Mobilität in folgenden Zügen. Für diese Metrik hat Michael Buro als aussagekräftigstes Merkmal die Summe aller
freien Felder ausgemacht, die um sämtliche gegnerischen Steine gezählt werden. Es wird also für jeden gegnerischen Stein
überprüft, an wie viele freie Felder dieser grenzt, und die Summe darüber gebildet.
\cite[S.~8f.]{evaluationfunctions}

\subsection{Cowthello}
\label{sec:cowthello}
Die Ecken haben in Othello eine große Bedeutung, da sie, nachdem sie besetzt sind, nicht mehr durch den Gegner umgedreht
werden können. Außerdem ist es wahrscheinlich, dass der Spieler weitere angrenzende Felder erobern kann, die dann
ebenfalls nicht mehr umgedreht werden können. Die Cowthello-Heuristik stammt aus dem gleichnamigen Online-Othello-Spiel
Cowthello und gewichtet die Spielfelder unterschiedlich.
\cite{cowthello}
Tabelle \ref{fig:cowthello} zeigt die Gewichtung der einzelnen Felder. Die Ecken sind hierbei am höchsten bewertet.
Felder vor den Ecken sind negativ bewertet, da sie es dem Gegner häufig ermöglichen, die Ecke zu besetzen. Die Felder
vor diesen angrenzenden Feldern sind wiederum positiv bewertet, da sie als Hilfsfelder die Möglichkeit bieten, die Ecke
später zu besetzen. Außerdem sind Randfelder höher bewertet, da sie nur durch ein weiteres Randfeld erobert werden
können.

\begin{figure}[H]
    \centering
    \includegraphics[width=\textwidth / 2]{cowthello}
    \caption{Gewichtung der Felder durch die Cowthello-Heuristik}
    \label{fig:cowthello}
\end{figure}

\subsubsection{Dynamische Gewichtung}
In Cowthello wird die Heuristik verändert, sobald Ecken belegt werden.
\cite{cowthello}

Das ist sinnvoll, da die an eine Ecke angrenzenden Felder negativ bewertet werden, weil sie dem Gegner Zugang zur Ecke
bieten. Wenn die Ecke bereits besetzt ist, spielt das keine Rolle mehr. Eine Möglichkeit, die Heuristik anzupassen,
besteht darin, die Felder vor einer Ecke mit dem Absolutwert zu gewichten, sobald die Ecke besetzt ist.
