\section{Heuristische Evaluations-Funktion}
\label{sec:heuristic}

%TODO: beschränkte tiefensuche separat erklären
In vielen relevanten Spielen ist es aufgrund der Größe des Suchbaums praktisch nicht möglich diesen vollständing zu
durchsuchen. In der Praxis wird deshalb häufig eine beschränkte Tiefensuche durchgeführt. An der maximalen Suchtiefe kommt
eine heuristische Evaluations-Funktion zum Einsatz, die den Nutzen des Spielzustands für einen Spieler approximiert.
\cite[S.~171]{ai2010russel}

Optimalerweise würde die Evaluations-Funktion den tatsächlichen Nutzen einer Spielsituation bestimmen. Eine solche
Evaluations-Funktion ist durch den Minimax Algorithmus gegeben. Aus offensichtlichen Gründen, kann diese jedoch nicht
zum Einsatz kommen, da der Rechenaufwand hierbei identisch zu einer vollständigen Suche wäre.

Um den Nutzen verschiedener Spielzüge miteinander zu vergleichen, wird also eine weniger rechenaufwändige
Evaluations-Funktion benötigt, die diesen nur ausreichend gut annähert, statt den tatsächlichen Nutzen zu bestimmen.

Durch die Anwendung einer approximierten Evaluations-Funktion verliert der Minimax Algorithmus seine Optimalität. Es
wird also nichtmehr garantiert der Zug mit dem größten Nutzen gewählt. Mit einer passend gewählten Evaluations-Funkion
können dennoch gute Ergebnisse erzielt werden, die in vielen Spielen die Fähigkeiten eines menschlichen Gegenspielers
übertreffen. Die Spielstärke der KI wächst mit zunehmender Suchtiefe.

Ein primitiver Ansatz für eine Evaluations-Funktion bei dem Spiel Othello ist die Differenz der Anzahlen von eigenen
Steinen und gegnerischen Steinen auf dem Spielfeld, da diese am Ende über den Sieg entscheidet. Gegen Ende eines Spiels
könnte dieser Ansatz eine passende Bewertung liefern, allerdings kann es vor allem in der Anfangsphase sinnvoll sein,
weniger Steine zu haben. Viel wichtiger ist z.\,B. die Möglichkeit, neue Steine zu platzieren.

Diesen Ansatz verfolgt die Betrachtung der Mobilität.

Die aktuelle Mobilität ist ein Maß dafür, wie viele mögliche Züge die Spieler haben. Eine niedrige Anzahl an möglichen
Zügen ist häufig schlecht, da der Spieler gezwungen ist, einen weniger guten Zug zu machen. Als quantitatives Merkmal
kann zunächst die Differenz der möglichen Züge beider Spieler betrachtet werden. Allerdings ist eine Stellung häufig
besser, wenn man mehr mögliche Züge hat und der Gegner bei gleicher Differenz weniger mögliche Züge hat. Um diese
Tatsache zu berücksichtigen, kann eine nicht-lineare Funktion verwendet werden.
%TODO: Buro verwendet aus Performancegründen eine Approximation
\cite[S. 7]{evaluationfunctions}

Neben der aktuellen Mobilität kann auch die potenzielle Mobilität betrachtet werden. Diese ist ein Indikator für die
mögliche Mobilität in folgenden Zügen. Michael Buro hat als aussagekräftigstes Merkmal die Summe aller Anzahlen freier
Felder um gegnerische Steine ausgemacht.
\cite[S. 8f.]{evaluationfunctions}
