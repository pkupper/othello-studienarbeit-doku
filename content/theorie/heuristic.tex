\section{Heuristische Evaluations-Funktion}

Um verschiedene Spielzüge miteinander zu vergleichen wird eine Evaluations-Funktion benötigt, die einen Spielzustand bewertet.
Unterschiedliche Funktionen führen zu unterschiedlich guten Ergebnissen.
Die Evaluations-Funktion ist also ein wichtiger Bestandteil einer KI.

Ein primitiver Ansatz für eine Evaluations-Funktion bei Othello ist die Differenz der Anzahlen von eigenen Steinen und gegnerischen Steinen auf dem Spielfeld, da diese am Ende über den Sieg entscheidet.
Gegen Ende eines Spiels könnte dieser Ansatz eine passende Bewertung liefern, allerdings kann es vor allem in der Anfangsphase sinnvoll sein, weniger Steine zu haben.
Viel wichtiger ist z.B. Möglichkeit, neue Steine zu platzieren.

Diesen Ansatz verfolgt die Betrachtung der Mobilität.

Die aktuelle Mobilität ist ein Maß dafür, wie viele mögliche Züge die Spieler haben.
Eine niedrige Anzahl an möglichen Zügen ist häufig schlecht, da der Spieler gezwungen ist, einen weniger guten Zug zu machen.
Als quantitatives Merkmal kann zunächst die Differenz der möglichen Züge beider Spieler betrachtet werden.
Allerdings ist eine Stellung häufig besser, wenn man mehr mögliche Züge hat und der Gegner bei gleicher Differenz weniger mögliche Züge hat.
Um diese Tatsache zu berücksichtigen, kann eine nicht-lineare Funktion verwendet werden.
%TODO: Buro verwendet aus Performancegründen eine Approximation
\cite[S. 7]{evaluationfunctions}

Neben der aktuellen Mobilität kann auch die potentielle Mobilität betrachtet werden.
Diese ist ein Indikatior für die mögliche Mobilität in folgenden Zügen.
Michael Buro hat als aussagekräftigstes Merkmal die Summe aller Anzahlen freier Felder um gegnerische Steine ausgemacht.
\cite[S. 8f.]{evaluationfunctions}

