\section{Beschränkte Tiefensuche}

In vielen relevanten Spielen ist es aufgrund der Größe des Suchbaums praktisch nicht möglich, diesen vollständing zu
durchsuchen. In der Praxis wird deshalb häufig eine beschränkte Tiefensuche durchgeführt. Dabei erhält die Rekursion
eine zusätzliche Abbruchbedingung. Neben dem Erreichen eines Endzustands wird auch bei der Erreichung einer
vorgegebenen Rekursionstiefe die Suche gestoppt.

Da in dem so erreichten Zustand der Ausgang des Spiels noch nicht bekannt ist, kann an dieser Stelle nicht der
tatsächliche Nutzen des Spielzustands zurückgegeben werden. Stattdessen wird eine Approximation des Nutzens verwendet \cite[S.~171]{ai2010russel}.

Diese Approximation wird im folgenden \autoref{sec:heuristic} genauer betrachtet.

Durch die Anwendung einer Approximation anstelle des tatsächlichen Nutzens verliert der Minimax-Algorithmus seine
Optimalität. Es wird also nicht mehr garantiert der Zug mit dem größten Nutzen gewählt. Mit einer passend gewählten
Approximation können dennoch gute Ergebnisse erzielt werden, die in vielen Spielen die Fähigkeiten eines menschlichen
Gegenspielers übertreffen.

Bei der beschränkten Tiefensuche wächst die Spielstärke der \ac{KI} in der Regel mit zunehmender Suchtiefe, da dadurch
Zustände betrachtet werden, die näher am Endzustand liegen.
