\section{Suchbäume}
\label{sec:gametree}

Ein Spiel kann als Baum dargestellt werden, dessen Wurzel gleich dem Startzustand des Spiels ist. Die Kanten des Baums
stellen Aktionen dar, die zu einem neuen Knoten, also einem neuen Spielzustand, führen. Vorerst sei angenommen, dass ein
bestimmter Spielzustand, in dem das Spiel gewonnen ist, gesucht wird und die Spielzüge des Gegners ausgesucht werden
können. Dieses Suchproblem hat als Lösung eine Abfolge von Aktionen, die zum Sieg des Spiels führen. Jeder Knoten kann
so erweitert werden, dass für jeden möglichen Spielzug, also jede Aktion, eine Kante zu einem neuen Knoten, also zu
einem neuen Spielzustand, führt. Der Baum kann dann nach einem bestimmten Zustand durchsucht werden, der das gegebene
Problem löst. Die Vorgehensweise dabei ist eine Tiefensuche. Zunächst wird eine Option bis zum Ende des Spiels
untersucht und erst wenn diese nicht zu einer Lösung des Problems führt, werden die übrigen Optionen in Betracht
gezogen.
\cite[S.~75]{ai2010russel}

\subsection{Adversative Suche}
In einem Spiel können die Züge des Gegners nicht beeinflusst werden. Deswegen ist ein einzelner Pfad von Aktionen
uninteressant und eine sogenannte Strategie, die für jeden Zug alle möglichen Reaktionen des Gegners betrachtet, ist
gesucht. Eine optimale Strategie führt unter der Annahme, dass auch der Gegner optimal spielt, zu einem mindestens
genauso guten Ergebnis wie jede andere Strategie.
\cite[S.~163f.]{ai2010russel}

Der Baum eines gesamten Spiels kann je nach Spiel sehr groß sein. Für das Spiel Othello besteht der Baum bei der Annahme
von einer durchschnittlichen Spiellänge von 58 Zügen und zehn Möglichkeiten pro Spielzug aus $10^{58}$ Knoten. Bei drei
verschiedenen Zuständen, die jedes Feld annehmen kann (frei, belegt von schwarz oder belegt von weiß), liegt die
Speicherkomplexität bei $3^{64}\approx10^{30}$ möglichen Spielzuständen. Diese Zahl lässt sich durch Ausschließen
offensichtlich ungültiger Zustände, in denen beispielsweise die mittleren vier Felder nicht belegt sind oder gesetzte
Steine nicht verbunden sind, auf ca. $10^{28}$ reduzieren \cite[S.~167]{searchingforsolutions}.
Bei einer Repräsentation von 2 Bit pro Feld, also 16 Byte pro Spielzustand, würden diese Zustände ca. $10^{20}$\,GB
Speicher benötigen. Der Spielbaum muss in diesem Fall also als theoretisches Konstrukt betrachtet werden, da er nicht
komplett dargestellt werden kann. Ein Suchbaum ist ein Teil des Spielbaums, der jedoch nur die für eine Entscheidung
relevanten Knoten beinhaltet.
\cite[S.~162f.]{ai2010russel}

Im allgemeinen Fall können Zustände mehrfach im Baum auftreten. Das passiert, wenn Spielzüge rückgängig gemacht werden
können. Beispielsweise können im Schach manche Figuren auf das vorherige Feld zurückgestellt werden. Für diese Spiele
ist es sinnvoll, zyklische Pfade auszuschließen. Dadurch wird verhindert, dass durch beliebig viele Wiederholungen eine
unendliche Menge an Pfaden entsteht.
\cite[S.~75]{ai2010russel}

In dem Spiel Othello kann allerdings kein Zustand mehrfach auftreten, da nach jedem Spielzug genau ein Spielstein mehr
auf dem Feld liegt und Spielsteine nicht wieder weggenommen werden können.
