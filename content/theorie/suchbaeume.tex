\section{Suchbäume}

%TODO: Aufteilung in Suchbäume allgemein und adversative Suche

Ein Suchproblem in einem Spiel hat als Lösung eine Abfolge von Aktionen, die zum Sieg des Spiels führen. Zur Darstellung
eines solchen Problems kann ein Baum verwendet werden, dessen Wurzel gleich dem Startzustand des Spiels ist. Die Kanten
des Baums stellen Aktionen dar, die zu einem neuen Knoten, also einem neuen Spielzustand, führen. Jeder Knoten kann so
erweitert werden, dass für jeden möglichen Spielzug, also jede Aktion, eine Kante zu einem neuen Knoten, also zu einem
neuen Spielzustand, führt. Vorerst sei angenommen, dass die Spielzüge des Gegners ausgesucht werden können. Der Baum
kann dann nach einem bestimmten Zustand durchsucht werden, der das gegebene Problem löst. Die Vorgehensweise dabei ist,
zunächst eine Option zu untersuchen und erst falls diese nicht zu einer Lösung des Problems führt, die übrigen Optionen
in Betracht zu ziehen.
\cite[S.~75]{ai2010russel}

In einem Spiel können die Züge des Gegners nicht beeinflusst werden. Deswegen ist ein einzelner Pfad von Aktionen
uninteressant, sondern eine sogenannte Strategie, die nach dem ersten Zug alle möglichen Reaktionen des Gegners
betrachtet.
\cite[S.~163]{ai2010russel}

Im allgemeinen Fall können Zustände mehrfach im Baum auftreten. Das passiert, wenn Spielzüge rückgängig gemacht werden
können, beispielsweise können im Schach manche Figuren auf das vorherige Feld zurückgestellt werden. Für diese Spiele
ist es sinnvoll, zyklische Pfade auszuschließen. %TODO: Graph statt Baum?
Dadurch wird verhindert, dass durch beliebig
viele Wiederholungen eine unendliche Menge an Pfaden entsteht.
\cite[S.~75]{ai2010russel}

In dem Spiel Othello kann allerdings kein Zustand mehrfach auftreten, da nach jedem Spielzug genau ein Spielstein mehr
auf dem Feld liegt und Spielsteine nicht wieder weggenommen werden können.

Der Baum eines gesamten Spiels kann je nach Spiel sehr groß sein. Für das Spiel Othello besteht der Baum bei der Annahme
von einer durchschnittlichen Spiellänge von 58 Zügen und zehn Möglichkeiten pro Spielzug aus $10^{58}$ Knoten. Bei drei
verschiedenen Zuständen, die jedes Feld annehmen kann (frei, belegt von schwarz oder belegt von weiß), liegt die
Speicherkomplexität bei $3^{64}\approx10^{30}$ möglichen Spielzuständen. Diese Zahl lässt sich durch Ausschließen
offensichtlich ungültiger Zustände, in denen beispielsweise die mittleren vier Felder nicht belegt sind oder gesetzte
Steine nicht verbunden sind, auf ca. $10^{28}$ reduzieren.
\cite[S.~167]{searchingforsolutions}
Bei einer Repräsentation von 2 Bit pro Spielfeld, also 16 Byte pro Zustand, würden diese Zustände ca. $10^{20}$\,GB
Speicher benötigen. Der Spielbaum muss in diesem Fall also als theoretisches Konstrukt betrachtet werden, da er nicht
komplett dargestellt werden kann. Ein Suchbaum ist ein Teil des Spielbaums, der jedoch nur die für eine Entscheidung
relevanten Knoten beinhaltet.
\cite[S.~162f.]{ai2010russel}
