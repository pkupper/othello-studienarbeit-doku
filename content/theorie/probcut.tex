\section{ProbCut}
\label{sec:probcut}

Der Minimax-Algorithmus erfordert in seiner Grundform das Durchsuchen des kompletten Spielbaums. Die Optimierung des
Algorithmus durch die Alpha-Beta-Suche ermöglicht zwar das Weglassen von Teilen des Spielbaums, aber nur dann, wenn das
Weglassen des Zweigs garantiert keine Auswirkung auf das Ergebnis des Algorithmus hat. ProbCut erweitert diese Idee
durch die Möglichkeit Zweige auszuschließen, welche mit einer großen Wahrscheinlichkeit das Ergebnis nicht beeinflussen.

Die mithilfe von ProbCut durchgeführten Verkleinerungen des Baums werden „probabilistic forward cuts“ genannt.

Ein Zug ist dann für die Entscheidung irrelevant, wenn dessen Minimax Wert außerhalb des Alpha-Beta Fensters, d. h.
außerhalb des Intervalls \([\alpha,\beta]\) liegt, da dieser Zug von beiden Spielern umgangen wird und deshalb im Spiel
nicht auftritt. Dies entspricht der Abbruchbedingung beim Alpha-Beta Algorithmus. Ob das der Fall ist, ist jedoch in der
Regel erst nach Anwendung des Alpha-Beta Algorithmus mit der maximal gewünschten Suchtiefe bekannt.

ProbCut stellt einen statistischen Zusammenhang zwischen dem Minimax-Wert einer Suche der Tiefe \(d\) und dem
Minimax-Wert einer flacheren Suche der Tiefe \(d'<d\) her. Anhand des Minimax-Wertes einer Suche der Tiefe \(d'\) wird
bestimmt, ob der Minimax-Wert der Suche der Tiefe \(d\) mit einer gegebenen Wahrscheinlichkeit \(p\) außerhalb des
Alpha-Beta Fensters liegt. Ist dies der Fall, so wird bereits nach der Suche der Tiefe \(d'\) abgebrochen, sodass die
zweite, tiefere Suche nicht mehr durchgeführt werden muss.

Dazu verwendet ProbCut ein statistisches Modell \(v=v'+e\), bei dem \(v\) der Minimax-Wert der Suche mit Tiefe \(d\),
\(v'\) der Minimax Wert der Suche mit Tiefe \(d'\) und \(e\) eine normalverteilte Fehlervariable ist. Der Minimax-Wert
der Suche der Tiefe \(d'\) weicht also von dem Minimax-Wert der Suche der Tiefe \(d\) um die Fehlervariable \(e\) ab.
Diese wird als normalverteilt mit dem Mittelwert $\mu=0$ und der Standardarbeichung \(\sigma\) modelliert, die anhand
von tatsächlichen Spielsituationen ermittelt wird.

Ob $v\geq\beta$ mit einer festgelegten Wahrscheinlichkeit stimmt, kann wie folgt bestimmt werden:

Einsetzen des statistischen Modells $v=v'+e$:

\hspace*{1.3cm}
$v\geq\beta\iff v'+e\geq\beta$

\hspace*{1.3cm}
$v\geq\beta\iff v'-\beta\geq -e$

\hspace*{1.3cm}
$v\geq\beta\iff \frac{v'-\beta}{\sigma}\geq \frac{-e}{\sigma}$

$\frac{-e}{\sigma}$ ist standardnormalverteilt und hat die Verteilungsfunktion $\Phi$, da der Mittelwert $\mu_1=0$ und
die Standardarbeichung $\sigma_1=\frac{\sigma}{\sigma}=1$ ist.

Dabei ist $\Phi(x)=\frac{1}{2\pi}\int_{-\infty}^xe^{-\frac{1}{2}t^2}$ Verteilungsfunktion der Standardnormalverteilung.
Es gilt also mit einer Wahrscheinlichkeit von mindestens $p$:

\hspace*{1.3cm}
$\frac{(v'-\beta)}{\sigma}\geq\Phi^{-1}(p)$

\hspace*{1.3cm}
$\iff v'\geq\Phi^{-1}(p)*\sigma+\beta$

Somit gilt auch mit einer Wahrscheinlichkeit von mindestens $p$:

\hspace*{1.3cm}
\(v\geq\beta \iff v'\geq\beta+\Phi^{-1}(p)*\sigma\)

Analog kann gezeigt werden, dass mit einer Wahrscheinlichkeit von mindestens $p$ folgende Formel gilt:

\hspace*{1.3cm}
\(v\leq\alpha \iff v'\leq\alpha-\Phi^{-1}(p)*\sigma\)


Tritt nach einer Suche der Tiefe \(d'\) einer der beiden Fälle auf der rechten Seite ein, so wird keine Suche der Tiefe
\(d\) mehr durchgeführt, da \(v\) bereits mit einer Wahrscheinlichkeit von \(p\) außerhalb des Alpha-Beta Fensters liegt
und der Zweig deshalb ausgeschlossen werden kann.

Wie auch beim Alpha-Beta Algorithmus ergibt sich der Vorteil von ProbCut durch die Zeitersparnis, die durch das
Weglassen von Zweigen erreicht wird. Dadurch kann die Suchtiefe bei den relevanten Zweigen erhöht werden.
\cite[S.~1]{probcut}
