\section{ProbCut}

Der Minimax-Algorithmus erfordert in seiner Grundform das Durchsuchen des kompletten Spielbaums.
Die Optimierung des Algorithmus durch die Alpha-Beta-Suche ermöglicht zwar das Weglassen von Teilen des Spielbaums, aber da auch hier der korrekte Minimax-Wert berechnet werden muss, kann der Baum nur
%TODO: "rückwärts" Erklären oder Entfernen
rückwärts reduziert werden.

ProbCut erweitert die Alpha-Beta-Suche durch eine statistische Betrachtung, wobei auch Zweige, die nur mit einer bestimmten Wahrscheinlichkeit nicht zu einer Veränderung des Ergebnisses beitragen, weggelassen werden können.
Die mithilfe von ProbCut durchgeführten Verkleinerungen des Baums werden "`probabilistic forward cuts"' genannt.

Ein Zug ist dann für die Entscheidung irrelevant, wenn dessen Minimax Wert außerhalb des Alpha-Beta Fensters, d.h. außerhalb des Intervalls \([\alpha,\beta]\) liegt, da dieser Zug von beiden Spielern umgangen wird und deshalb im Spiel nicht auftritt.
Dies entspricht der Abbruchbedingung beim Alpha-Beta Algorithmus.
Ob das der Fall ist, ist jedoch in der Regel erst nach Anwendung des Alpha-Beta Algorithmus' mit der maximal gewünschten Suchtiefe bekannt.

ProbCut stellt einen statistischen Zusammenhang zwischen dem Minimax Wert einer tieferen Suche der Tiefe \(d\) und dem Wert einer flacheren Suche der Tiefe \(d'<d\) her.
Anhand des Minimax Werts der flachen Suche wird bestimmt ob der Wert der tiefen Suche mit einer gegebenen Wahrscheinlichkeit \(p\) außerhalb des Alpha-Beta Fensters liegt.
Ist dies der Fall, so wird bereits nach der flachen Suche abgebrochen, sodass die Tiefe Suche nicht mehr durchgeführt werden muss.

%TODO: a und b entfernen (Ersetzen mit 1 und 0)
Dazu verwendet ProbCut ein lineares statistisches Modell \(v=a*v'+b+e\), wobei \(v\) der Minimax Wert der tiefen Suche, \(v'\) der Minimax Wert der flachen Suche, \(a\) und \(b\) reelle Konstanten und \(e\) eine normalverteilte Fehlervariable ist.
\(a\) und \(b\) werden im Vorraus so gewählt, dass die Standardabweichung \(\sigma^{2}\) von \(e\) möglichst gering ist.

Daraus lässt sich herleiten, dass mit einer Wahrscheinlichkeit von mindestens \(p\) gilt:

\(v\leq\alpha \iff v'\leq\frac{\Phi^{-1}(p)*\sigma+\alpha-b}{a}\)

und

\(v\geq\beta\ \iff v'\geq\frac{-\Phi^{-1}(p)*\sigma+\beta-b}{a}\)

Trifft nach der flachen Suche einer der beiden Fälle auf der rechten Seite ein, so wird keine tiefe Suche durchgeführt, da \(v\) nach der linken Seite mit Wahrscheinlichkeit \(p\) außerhalb des Alpha-Beta Fensters liegt
und somit nicht relevant ist.

Wie auch beim Alpha-Beta Algorithmus ergibt sich der Vorteil von ProbCut durch die Zeitersparnis, die durch das Weglassen von Zweigen erreicht wird. Dadurch kann die Suchtiefe bei den relevanten Zweigen erhöht werden.
\cite[S.~1]{probcut}
