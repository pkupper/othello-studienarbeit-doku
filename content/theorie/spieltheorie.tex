\section{Spieltheorie}

Eine mathematische Betrachtung von Spielen erfordert die abstrakte Darstellung von Spielen.
Ein Spiel kann dabei verschiedene Eigenschaften haben. Viele Spiele weisen folgende Eigenschaften auf:

\begin{itemize}
    \item Deterministisch
    \item Rundenbasiert
    \item Zwei Spieler
    \item Nullsummenspiel
    \item Perfekte Information
\end{itemize}

Diese Eigenschaften treffen auch auf das Spiel Othello zu:

Die Eigenschaft des Determinismus schließt auf Zufall basierende Elemente aus, die beispielsweise in Würfelspielen durch das Würfeln einer zufälligen Zahl auftreten.
Da in Othello jeder Spielzug nur von der Entscheidung des Spielers abhängt, wird diese Eigenschaft erfüllt.
Im Spielverlauf sind weiterhin zwei Spieler abwechselnd am Zug.
Bei einem Nullsummenspiel handelt es sich um ein Spiel, bei dem das Ergebnis beider Spieler am Ende jedes Spiels die gleiche Summe hat.
Beim Schach bekommt z.B. der Gewinner einen Punkt und der Verlierer keinen Punkt, bei einem Unentschieden bekommen beide Spieler einen halben Punkt.
Die Summe beträgt also immer eins.
Diese Regel kann auch in Othello angewandt werden, da ein Spiel etweder mit einem Sieg und einer Niederlage oder einem Unentschieden endet.
Um die Höhe des Sieges zu berücksichtigen, kann für jeden Spieler die Anzahl der Steine gezählt werden.
Dabei müssen lediglich nicht belegte Felder berücksichtigt und beispielsweise für den Gewinner mit mehr Steinen gezählt werden.
Da sich die Anzahl der Gesamtfelder nicht ändert, beträgt die Summe der Punkte beider Spieler bei einer Spielfeldgröße von 8x8 immer 64 und somit handelt es sich bei Othello um ein Nullsummenspiel.
Bei einem Spiel mit perfekter Information können beide Spieler zu jeder Zeit den kompletten Zustand des Spiels einsehen.
Anders als bei vielen Kartenspielen, in denen die Karten der Gegenspieler unbekannt sind, ist das in Othello der Fall.
\cite[S.~161f.]{ai2010russel}

Formal besteht ein Spiel aus folgenden sechs Elementen:
\begin{enumerate}
    \item $S_0$ ist der Anfangszustand zu Beginn des Spiels.
    \item $Player(s)$ ist eine Funktion, die einen Spielzustand $s$ den Spieler berechnet, der gerade an der Reihe ist.
    \item $Actions(s)$ ist eine Funktion, die für einen Spielzustand $s$ alle gültigen Züge berechnet.
    \item $Result(s, a)$ ist eine Funktion, die für einen Spielzustand $s$ und eine Aktion, also einen Spielzug $a$ einen neuen Spielzustand berechnet.
    \item $TerminalTest(s)$ ist eine Funktion, die für einen Spielzustand $s$ berechnet, ob das Spiel vorbei ist oder nicht. Zustände, an denen das Spiel vorbei ist, werden Endzustände genannt.
    \item $Utility(s, p)$ ist eine Funktion, die für einen Endzustand $s$ ein numerisches Endergebnis für den Spieler $p$ berechnet.
\end{enumerate}
