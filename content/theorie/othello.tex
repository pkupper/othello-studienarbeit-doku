\section{Othello}

Othello ist ein Spiel für zwei Spieler, in dem es darum geht, zum Ende des Spiels möglichst viele Steine der eigenen Farbe auf dem Spielfeld, einem 8x8 Felder großem quadratischen Gitter, liegen zu haben.

Zu Beginn des Spiels liegen auf den vier mittleren Feldern des Spielfelds jeweils zwei Steine jedes Spielers so, dass die Steine eines Spielers einander diagonal gegenüber liegen.
Die Spieler legen nun abwechselnd jeweils einen Stein ihrer eignen Farbe auf das Spielfeld.

Steine können nur auf freie Felder gelegt werden, die in horizontale, vertikale oder diagonale Richtung an einen oder mehrere gegnerische Steine, gefolgt von einem eigenen Stein angrenzen.

Alle gegnerischen Steine, die durch den gesetzten Stein in eine der 8 Richtungen lückenlos, d.h. ohne freie Felder dazwischen, eingeschlossen werden, werden umgedreht und ändern dadurch ihre Farbe zu der des anderen Spielers.

Ist einer der Spieler zugunfähig, das heißt, er ist an der Reihe jedoch kann er keinen validen Zug spielen, so ist der andere Spieler erneut am Zug.

Das Spiel endet, sobald beide Spieler zugunfähig sind. Der Gewinner ist dann der Spieler, der die meisten Steine seiner Farbe auf dem Spielfeld liegen hat.
\cite{worldothellorules}
