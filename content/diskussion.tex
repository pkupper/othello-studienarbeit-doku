%!TEX root = ../dokumentation.tex

\chapter{Diskussion}
\label{chap:diskussion}
Dieses Kapitel reflektiert die im Rahmen dieser Arbeit erzielten Ergebnisse und zeigt dabei die Stärken und
Schwachstellen der KI auf.

\section{Reflexion}
Wie im vorigen Kapitel gesehen, ist die in dieser Arbeit entwickelte künstliche Intelligenz in der Lage, gegen mehrere
online verfügbare KIs anzutreten und zeigt dabei herausragende Ergebnisse. Damit erfüllt sie das in \label{sec:goal}
festgelegte Ziel.

Dennoch besteht ein großes Potenzial für Verbesserungen und Optimierungen. Das größe Performance-Bottleneck besteht, wie
zu erwarten ist, in der Berechnung der Heuristik. Insgesamt wird ca. \(67\%\) der Zeit dafür aufgewandt. Durch
Optimierungen an dieser Stelle kann daher ein großer Performancegewinn erzielt werden, wodurch eine höhere Suchtiefe
erzielbar wird. So gelang es Buro beispielsweise, in der KI Logistello eine Verbesserung der Ausführungsgeschwindigkeit
um den Faktor 3 durch Verwendung einer approximierten Mobilität zu erlangen, wobei die Stärke der KI nur minimal
beeinflusst wurde. \cite[S.~8]{evaluationfunctions}

Durch die Verwendung einer kompilierten, statt einer interpretierten Programmiersprache kann die Performanz potenziell
wesentlich gesteigert werden.


\section{Ausblick}
Die Gewichtung von Mobilität und Cowthello-Heuristik wurde nur auf Vielfache von 12.5\,\% genau bestimmt. Da die
Unterschiede in der Stärke der KI jedoch immer geringer werden, ist eine genauere Bestimmung nur mit einer größeren
Menge an Daten möglich, dessen Sammlung außerhalb des zeitlichen Rahmens dieser Arbeit liegt. In weiteren Untersuchungen
könnte diese Gewichtung genauer bestimmt werden. Dadurch ließe sich potenziell eine geringe Verbesserung der Spielstärke
erzielen.

Ähnliches gilt für die Gewichtung der aktuellen und potenziellen Mobilität. Auch hier wurden nur wenige Alternativen
gegeneinander ausgespielt, um zum Ergebnis einer linearen Kombination der Merkmale zu kommen. Eine tiefergehende
Untersuchung kann hier möglicherweise zu einem besseren Ergebnis kommen.