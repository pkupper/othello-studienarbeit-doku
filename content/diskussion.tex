%!TEX root = ../dokumentation.tex

\chapter{Diskussion}
\label{chap:diskussion}
Dieses Kapitel reflektiert die im Rahmen dieser Arbeit erzielten Ergebnisse und zeigt dabei die Stärken und
Schwachstellen der \ac{KI} auf.

\section{Reflexion}
Wie im vorigen Kapitel gesehen, ist die in dieser Arbeit entwickelte \ac{KI} in der Lage, gegen mehrere
online verfügbare \acp{KI} anzutreten und zeigt dabei herausragende Ergebnisse. Damit erfüllt sie das in \autoref{sec:goal}
festgelegte Ziel.

Die dabei verwendeten Optimierungen ProbCut und iterative Tiefensuche zeigen jedoch nicht den gewünschten Effekt.
Obwohl in beiden Fällen die Heuristik meist seltener ausgewertet werden muss als ohne die Optimierung, gleicht dies jedoch
nicht den Rechenaufwand der zusätzlichen Suchen aus.

Großes Potenzial für eine Verbesserung der Performanz liegt zudem in der Heuristik. Mit einem Zeitaufwand von ca.
\(67\%\) stellt sie das größte Bottleneck der \ac{KI} dar. Durch Optimierungen an dieser Stelle kann daher ein großer
Performancegewinn erzielt werden, wodurch eine höhere Suchtiefe erzielbar wird. So gelang es Buro beispielsweise, in der
\ac{KI} Logistello eine Verbesserung der Ausführungsgeschwindigkeit um den Faktor 3 durch Verwendung einer approximierten
Mobilität zu erlangen, wobei die Stärke der \ac{KI} nur minimal beeinflusst wurde. \cite[S.~8]{evaluationfunctions}

Zudem kann durch die Verwendung einer kompilierten, statt einer interpretierten Programmiersprache die Performanz
ebenfalls potenziell wesentlich gesteigert werden.


\section{Ausblick}
Die Gewichtung von Mobilität und Cowthello-Heuristik wurde nur auf Vielfache von \(12.5\%\) genau bestimmt. Da die
Unterschiede in der Stärke der \ac{KI} jedoch immer geringer werden, ist eine genauere Bestimmung nur mit einer größeren
Menge an Daten möglich, dessen Sammlung außerhalb des zeitlichen Rahmens dieser Arbeit liegt. In weiteren Untersuchungen
könnte diese Gewichtung genauer bestimmt werden. Dadurch ließe sich potenziell eine geringe Verbesserung der Spielstärke
erzielen.

Ähnliches gilt für die Gewichtung der aktuellen und potenziellen Mobilität. Auch hier wurden nur wenige Alternativen
gegeneinander ausgespielt, um zum Ergebnis einer linearen Kombination der Merkmale zu kommen. Eine tiefergehende
Untersuchung kann hier möglicherweise zu einem besseren Ergebnis kommen.

Eine andere Wahl der ProbCut Parameter könnte zudem zu einer Verbesserung der Effektivität dieser Optimierung führen.
Insbesondere die Verwendung eines größeren Intervalls zwischen den beiden Suchtiefen \(d\) und \(d'\) im ProbCut
Algorithmus könnte sich positiv auswirken. Da die insgesamt erreichte Suchtiefe der \ac{KI} in einer angemessenen Zeit
nur bei 5 bis 8 liegt, ist dies jedoch schwer zu realisieren.

