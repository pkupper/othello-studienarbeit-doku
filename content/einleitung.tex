%!TEX root = ../dokumentation.tex

\chapter{Einleitung}
\label{chap:einleitung}

Diese Arbeit behandelt die Implementierung einer \acp{KI} für das Brettspiel Othello. Dabei kommt
ProbCut\cite[S.~1]{probcut}, eine Erweiterung des Alpha-Beta Algorithmus zum Einsatz.

\section{Ziel der Arbeit}
Das Ziel der Arbeit ist es, eine möglichst starke Künstliche Intelligenz für das Spiel Othello zu entwickeln, welche
gegen menschliche Spieler, sowie einige öffentlich verfügbare KI-Agenten gewinnt.

Zur Implementierung werden Jupyter Notebooks, sowie die Programmiersprache Python genutzt.
Für menschliche Spieler wird eine grafische Oberfläche zur Verfügung gestellt, über
die dieser gegen die Künstliche Intelligenz antreten kann.

\section{Struktur der Arbeit}
Das \autoref{chap:theorie} stellt zunächst das Spiel Othello und dessen Spielregeln vor. Im Anschluss werden die
relevanten theoretischen Grundlagen der Spieltheorie, sowie die für die Implementierung relevanten Algorithmen
vorgestellt und erklärt.

\autoref{chap:methode} beschreibt die Vorgehensweise bei der Implementierung und geht insbesondere darauf ein, wie
verschiedene Varianten der KI gegeneinander getestet und dadurch bewertet werden.

\autoref{chap:implementation} beinhaltet die Implementation der Künstlichen Intelligenz als Jupyter Notebook. Parallel
zur Implementation wird deren Funktionsweise ausführlich erklärt.

Das \autoref{chap:merkmale} behandelt die Bestimmung optimaler Parameter für die \acp{KI}. Dazu gehören die in der
Heuristischen-Evaluationsfunktion verwendeten Merkmale und deren Gewichtung. Die Messung der Auswirkungen von
AlphaBeta-Pruning und ProbCut, sowie die Bestimmung der ProbCut Parameter.

In \autoref{chap:ergebnis} werden die Stärke der KI, sowie die Auswirkungen von ProbCut untersucht.

In \autoref{chap:diskussion} werden die Ergebnisse der Arbeit reflektiert bewertet. Außerdem werden Schlüsse bezüglich
Problempunkten und potenziellen Weiterentwicklungen gezogen.