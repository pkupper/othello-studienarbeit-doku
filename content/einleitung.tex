%!TEX root = ../dokumentation.tex

\chapter{Einleitung}
\label{chap:einleitung}

Diese Arbeit behandelt die Implementierung einer Künstlichen Intelligenz (\acs{KI}) für das Brettspiel Othello. Dabei kommt ProbCut
\cite[S.~1]{probcut}, eine Erweiterung des Alpha-Beta-Algorithmus, zum Einsatz.

Die \LaTeX-Quellen dieser Arbeit sind auf GitHub unter \url{https://github.com/pkupper/othello-studienarbeit-doku}
verfügbar. Die Implementierung ist unter \url{https://github.com/pkupper/othello-studienarbeit-code} zu finden.

\section{Ziel der Arbeit}
\label{sec:goal}

Das Ziel der Arbeit ist es, eine möglichst starke \ac{KI} für das Spiel Othello zu entwickeln, die sowohl
gegen menschliche Spieler als auch gegen einige öffentlich verfügbare \ac{KI}-Agenten gewinnt.

Zur Implementierung werden Jupyter Notebooks sowie die Programmiersprache Python genutzt.
Für menschliche Spieler wird eine \ac{GUI} zur Verfügung gestellt, über
die diese gegen die \ac{KI} antreten können.

\section{Struktur der Arbeit}
In \autoref{chap:theorie} werden zunächst das Spiel Othello und dessen Spielregeln erläutert. Im Anschluss werden die
relevanten theoretischen Grundlagen der Spieltheorie, sowie die für die Implementierung relevanten Algorithmen
vorgestellt und erklärt.

\autoref{chap:methode} beschreibt die Vorgehensweise bei der Implementierung und geht insbesondere darauf ein, wie
verschiedene Varianten der \ac{KI} gegeneinander getestet und dadurch bewertet werden.

\autoref{chap:implementation} beinhaltet die Implementierung der \ac{KI} als Jupyter Notebook. Parallel
zur Implementierung wird die Funktionsweise der \ac{KI} ausführlich erklärt.

\autoref{chap:merkmale} behandelt die Bestimmung optimaler Parameter für die \ac{KI} und untersucht die
verschiedenen Strategien. Dazu gehören die in der heuristischen Evaluationsfunktion verwendeten Merkmale und deren
Gewichtung, die Messung der Auswirkungen von Alpha-Beta-Pruning und ProbCut, die Überprüfung der Korrektheit von
ProbCut sowie die Bestimmung der ProbCut-Parameter.

In \autoref{chap:ergebnis} werden die Stärke der \ac{KI} und die Auswirkungen von ProbCut untersucht.

In \autoref{chap:diskussion} werden die Ergebnisse der Arbeit reflektiert bewertet. Außerdem werden Schlüsse bezüglich
Problempunkten und potenziellen Weiterentwicklungen gezogen.
