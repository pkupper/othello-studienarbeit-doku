%!TEX root = ../dokumentation.tex

\chapter{Methode}
\label{chap:methode}

Ziel dieser Arbeit ist es, eine starke KI zu entwickeln. Dabei soll der Code verständlich und leicht nachvollziehbar
sein. Aus diesem Grund wird die KI in der Programmiersprache Python in Jupyter Notebooks implementiert. Diese
ermöglichen es, Dokumentation und Erklärungen direkt in Textfeldern beim entsprechenden Code zu schreiben.

Die KI soll auf dem Minimax-Algorithmus mit begrenzter Tiefensuche und den Erweiterungen Alpha-Beta und ProbCut
basieren.

Um eine möglichst starke KI zu entwickeln, muss überprüft werden, welche Merkmale wie stark in die Heuristik einfließen.
Dazu wird die Alpha-Beta-KI gegen sich selbst mit verschiedenen Heuristiken spielen gelassen. Beide KIs machen gleich
oft den ersten Zug, um auszuschließen, dass durch das Beginnen ein Vorteil oder Nachteil entsteht. Es wird der
Alpha-Beta-Algorithmus verwendet, da die Standardabweichung des ProbCut-Algorithmus von der Heuristik abhängt und jedes
Mal neu bestimmt werden müsste. Die Spiele verlaufen nicht deterministisch, da nicht unbedingt der Zug mit dem höchsten
Nutzen gewählt wird, sondern ggf. auch ein Zug, dessen Nutzen sehr ähnlich ist. Die Auswahl geschieht zufällig.

Kapitel \ref{chap:implementation} enthält die Implementation von Jupyter Notebooks, welche die KI, eine grafische
Oberfläche und Funktionalitäten zum Testen, Bestimmen und Evaluieren von Parametern implementieren.
