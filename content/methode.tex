%!TEX root = ../dokumentation.tex

\chapter{Methode}
\label{chap:methode}

Dieses Kapitel beschreibt die Vorgehensweise bei der Implementierung, Optimierung und beim Testen der \ac{KI}.

\section{Implementierung}
Ziel dieser Arbeit ist es, eine starke \ac{KI} zu entwickeln. Dabei soll der Code verständlich und leicht
nachvollziehbar sein. Aus diesem Grund wird die \ac{KI} in der Programmiersprache Python in Jupyter Notebooks
implementiert. Jupyter Notebooks ermöglichen es, Dokumentation und Erklärungen direkt in Textfeldern beim entsprechenden
Code zu schreiben.

Die \ac{KI} soll auf dem Minimax-Algorithmus mit begrenzter Tiefensuche und den Erweiterungen Alpha-Beta und ProbCut
basieren.

Kapitel \ref{chap:implementation} enthält die Jupyter Notebooks, welche die \ac{KI}, eine \ac{GUI} und Funktionalitäten
zum Testen, Bestimmen und Evaluieren von Parametern implementieren.

\section{Optimierung der Parameter}
\label{sec:optimization}
Um eine möglichst starke \ac{KI} zu entwickeln, muss überprüft werden, welche Merkmale wie stark in die Heuristik
einfließen. Zu diesem Zweck lässt man die Alpha-Beta-\ac{KI} unter Verwendung verschiedener Heuristiken gegen sich
selbst spielen. Beide Varianten der \ac{KI} machen gleich oft den ersten Zug, um auszuschließen, dass durch das Beginnen
ein Vorteil oder Nachteil entsteht. Es wird der Alpha-Beta-Algorithmus verwendet, da die Standardabweichung des
ProbCut-Algorithmus von der Heuristik abhängt und jedes Mal neu bestimmt werden müsste. Die Spiele verlaufen nicht
deterministisch, da nicht unbedingt der Zug mit dem höchsten Nutzen gewählt wird, sondern ggf. auch ein Zug, dessen
Nutzen sehr ähnlich ist. Die Auswahl zwischen geeigneten Zügen geschieht dabei zufällig.

\section{Test gegen andere KIs}
Beim Testen der \ac{KI} gegen andere \acp{KI} wird ähnlich wie in \autoref{sec:optimization} vorgegangen. Die zu
vergleichenden \acp{KI} treten mehrmals geneneinander an. Hierbei wird jedoch der ProbCut-Algorithmus in Kombination mit
der Heurisik verwendet, die den Untersuchungen in \autoref{sec:optimization} nach am stärksten ist. Diese Tests können
entweder manuell durchgeführt werden, indem die Züge einer \ac{KI} in das Spiel gegen die andere \ac{KI} übertragen
werden, oder lassen sich alternativ durch Code automatisieren. Eine solche Automatisierung wird für den Vergleich mit den, in
einer vorangegangenen Studienarbeit entwickelten, \acp{KI} verwendet.
