%!TEX root = ../dokumentation.tex

\chapter{Methode}
\label{chap:methode}

Ziel dieser Arbeit ist es, eine starke KI zu entwickeln. Dabei soll der Code verständlich und leicht nachvollziehbar
sein. Aus diesem Grund wird die KI in der Programmiersprache Python in Jupyter Notebooks implementiert. Diese
ermöglichen es, Dokumentation und Erklärungen direkt in Textfeldern beim entsprechenden Code zu schreiben.

Die KI soll auf dem MiniMax-Algorithmus mit begrenzter Tiefensuche und den Erweiterungen Alpha-Beta und ProbCut
basieren.

Kapitel \ref{chap:implementation} enthält die Implementation von Jupyter Notebooks, welche die KI, eine grafische Oberfläche und Funktionalitäten zum Testen, Bestimmen und Evaluieren von Parametern implementieren.
