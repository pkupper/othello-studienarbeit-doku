\hypertarget{gui-implementierung-othello_gui.ipynb}{%
\section{GUI Implementierung
(othello\_gui.ipynb)}\label{gui-implementierung-othello_gui.ipynb}}

\label{sec:gui} \ifx false

\begin{lstlisting}[language=Python]
%%HTML
<style>
.container { width:100% }
</style>
\end{lstlisting}

\fi Im folgenden Abschnitt wird eine \ac{GUI} für das Spiel Othello
implementiert. Diese ermöglicht das Anzeigen Spielfelds und das Setzen
von Steinen durch einen menschlichen Spieler per Mausklick. Außerdem
wird ein Einstellungsmenü implementiert, über welches das Spiel und die
\ac{KI}-Agenten konfiguriert werden können.

\hypertarget{importieren-der-externen-abhuxe4ngigkeiten}{%
\subsection{Importieren der externen
Abhängigkeiten}\label{importieren-der-externen-abhuxe4ngigkeiten}}

Die \ac{GUI} verwendet zur Darstellung des Spielzustandes, zum Anzeigen
erweiterter Informationen sowie für die Benutzerinteraktion die
Bibliotheken \passthrough{\lstinline!ipycanvas!} und
\passthrough{\lstinline!ipywidgets!}. Diese können direkt im Jupyter
Notebook verwendet werden.

Zusätzlich werden aus dem Paket \passthrough{\lstinline!math!} der
Python Standardbibliothek die Variable \passthrough{\lstinline!pi!}
sowie die Funktion \passthrough{\lstinline!floor!} benötigt.

\begin{lstlisting}[language=Python]
import ipycanvas
import ipywidgets
import math
from ipywidgets import RadioButtons, HBox, VBox, IntSlider, Label
\end{lstlisting}

\hypertarget{konfiguration-der-gui}{%
\subsection{Konfiguration der GUI}\label{konfiguration-der-gui}}

Die \ac{GUI} unterstützt optional die Anzeige aller möglichen Züge in
der aktuellen Spielsituation, sowie die Darstellung der Menge
\passthrough{\lstinline!frontier!}, welche im \autoref{sec:gamelogic}
beschrieben wird. Ob diese Features verwendet werden wird im Folgenden
mit den Konstanten \passthrough{\lstinline!SHOW\_FRONTIER!} und
\passthrough{\lstinline!SHOW\_POSSIBLE\_MOVES!} konfiguriert.

\begin{lstlisting}[language=Python]
SHOW_FRONTIER = False
SHOW_POSSIBLE_MOVES = True
\end{lstlisting}

\hypertarget{canvas-initialisierung}{%
\subsection{Canvas Initialisierung}\label{canvas-initialisierung}}

Zunächst wird ein Canvas Objekt der Bibliothek
\passthrough{\lstinline!ipycanvas!} initialisiert, auf dem später das
Spielfeld dargestellt wird. Die Konstante
\passthrough{\lstinline!CELL\_SIZE!} beschreibt hierbei die Größe eines
einzelen Feldes in Pixeln auf dem Spielbrett, wohingegen
\passthrough{\lstinline!CANVAS\_SIZE!} die Größe des gesamten
Spielbretts angibt.

\begin{lstlisting}[language=Python]
CELL_SIZE = 60

CANVAS_SIZE = BOARD_SIZE * CELL_SIZE

canvas = ipycanvas.MultiCanvas(2, width=CANVAS_SIZE, height=CANVAS_SIZE)
canvas[0].fill_style = 'darkgreen'
canvas[0].stroke_style = 'black'
canvas[0].fill_rect(0, 0, CANVAS_SIZE, CANVAS_SIZE)
canvas[0].begin_path()
for i in range(BOARD_SIZE+1):
    pos = i * CELL_SIZE
    canvas[0].move_to(pos, 0)
    canvas[0].line_to(pos, CANVAS_SIZE)
    canvas[0].move_to(0, pos)
    canvas[0].line_to(CANVAS_SIZE, pos)
canvas[0].stroke()
\end{lstlisting}

\hypertarget{widget-initialisierung}{%
\subsection{Widget Initialisierung}\label{widget-initialisierung}}

Unter dem Spielfeld werden mit der Bibliothek
\passthrough{\lstinline!ipywidgets!} zusätzliche Informationen zum
aktuellen Spielzustand angezeigt. Diese werden mithilfe sogenannter
Widgets dargestellt, welche im Folgenden erzeugt werden.

Das \passthrough{\lstinline!score\_lbl!} Widget enthält die Steinzahl
beider Spieler im aktuellen Spielzustand.

\begin{lstlisting}[language=Python]
score_lbl = ipywidgets.widgets.Label()
\end{lstlisting}

Das \passthrough{\lstinline!turn\_lbl!} Widget nennt den Spieler, der
gerade am Zug ist.

\begin{lstlisting}[language=Python]
turn_lbl = ipywidgets.widgets.Label()
\end{lstlisting}

Das \passthrough{\lstinline!output!} Widget macht die Ausgabe mithilfe
von \passthrough{\lstinline!print()!} sowie die Ausgabe von
Fehlermeldungen trotz der Verwendung von IPyWidgets und IPyCanvas
möglich.

\begin{lstlisting}[language=Python]
output = ipywidgets.widgets.Output()
\end{lstlisting}

Das Widget \passthrough{\lstinline!utility\_lbl!} dient der Anzeige der
von den \ac{KI} Spielern geschätzten Nützlichkeit des Spielzustands.

\begin{lstlisting}[language=Python]
utility_lbl = ipywidgets.widgets.Label()
\end{lstlisting}

In der Funktion \passthrough{\lstinline!update\_output!} wird der
Spielzustand \passthrough{\lstinline!state!} auf den existierenden
Canvas gezeichnet.

\begin{lstlisting}[language=Python]
DISK_RADIUS = 0.45
HINT_RADIUS = 0.2

def update_output(state):
    with ipycanvas.hold_canvas(canvas):
        canvas[1].clear()
        for ((x, y), val) in np.ndenumerate(state.board):
            if val == NONE:
                continue
            elif val == BLACK:
                canvas[1].fill_style = 'black'
            else:
                canvas[1].fill_style = 'white'
            canvas[1].fill_arc((x+0.5) * CELL_SIZE, (y+0.5)
                        * CELL_SIZE, CELL_SIZE*DISK_RADIUS, 0, 2*math.pi)

        if state.last_move != None:
            (x, y) = state.last_move
            canvas[1].stroke_style = 'red'
            canvas[1].line_width = 2
            canvas[1].stroke_arc((x+0.5) * CELL_SIZE, (y+0.5)
                        * CELL_SIZE, CELL_SIZE*DISK_RADIUS, 0, 2*math.pi)

        if SHOW_FRONTIER:
            for (x, y) in state.frontier:
                canvas[1].fill_style = 'gray'
                canvas[1].fill_arc((x+0.5) * CELL_SIZE, (y+0.5)
                        * CELL_SIZE, CELL_SIZE*HINT_RADIUS, 0, 2*math.pi)

        if SHOW_POSSIBLE_MOVES:
            for (x, y) in get_possible_moves(state, state.turn):
                if state.turn == BLACK:
                    canvas[1].fill_style = 'black'
                else:
                    canvas[1].fill_style = 'white'
                canvas[1].fill_arc((x+0.5) * CELL_SIZE, (y+0.5)
                        * CELL_SIZE, CELL_SIZE*HINT_RADIUS, 0, 2*math.pi)

    b_score = count_disks(state, BLACK)
    w_score = count_disks(state, WHITE)
    score_lbl.value = f'Black Player : {b_score} White Player : {w_score}'
    utility_lbl.value = 'Utility: Black: {:.4f} / White: {:.4f}'.format(
        utilities[BLACK], utilities[WHITE])
    if state.game_over:
        turn_lbl.value = f'{get_player_string(get_utility(state))} wins'
    else:
        turn_lbl.value = f'{get_player_string(state.turn)}s Move'
\end{lstlisting}

Die Funktion \passthrough{\lstinline!display\_board!} stellt den durch
\passthrough{\lstinline!state!} angegebenen Spielzustand dar, indem
zunächst der Canvas per \passthrough{\lstinline!update\_output!}
aktualisiert, und dann zusammen mit den Status-Widgets angezeigt wird.

\begin{lstlisting}[language=Python]
def display_board(state):
    output.clear_output()
    update_output(state)
    display(canvas)
    display(score_lbl)
    display(turn_lbl)
    display(utility_lbl)
    display(output)
\end{lstlisting}

Für menschliche Spieler muss festgestellt werden, ob auf das Spielfeld
geklickt wurde. Dies geschieht in der Callback Funktion
\passthrough{\lstinline!mouse\_down!}, welche die x und y Koordinate des
Mausklicks relativ zum Canvas erhält. Auf Basis dieser Position wird,
falls möglich, ein Zug auf das angeklickte Feld gemacht. Die Funktion
wird durch den Aufruf von \passthrough{\lstinline!on\_mouse\_down!} auf
dem IPyCanvas als Callback Funktion registriert.

\begin{lstlisting}[language=Python]
def mouse_down(x_px, y_px):
    global state
    with output:
        if not state.game_over:
            x = math.floor(x_px / CELL_SIZE)
            y = math.floor(y_px / CELL_SIZE)
            try:
                state = make_move(state, (x, y))
            except InvalidMoveException:
                print('Invalid Move')
            update_output(state)
            try:
                next_move(state)
            except KeyboardInterrupt:
                pass

canvas[1].on_mouse_down(mouse_down)
\end{lstlisting}

\hypertarget{spieleinstellungen}{%
\subsection{Spieleinstellungen}\label{spieleinstellungen}}

Hier wird durch Nutzung von \passthrough{\lstinline!ipywidgets!} eine
\ac{GUI} erzeugt, über die für beide Spieler Einstellungen vorgenommen
werden können. So kann zum Beispiel festelegt werden, ob ein Spieler vom
Nutzer, oder von einem \ac{KI}-Agenten kontrolliert werden soll und
Parameter der \ac{KI} angepasst werden.

\begin{lstlisting}[language=Python]
algorithms = { 'Menschlicher Spieler': None,
               'ProbCut': probcut,
               'Alpha-Beta': alphabeta,
               'Minimax': minimax,
               'Zufällig': random_ai }
modes = { 'Feste Tiefe': ai_make_move,
          'Iterative Vertiefung': ai_make_move_id,
          'Zeitbegrenzte Vertiefung': ai_make_move_id_timelimited }
heuristics = { 'Cowthello': cowthello_heuristic,
               'Mobilität': mobility_heuristic,
               'Kombiniert': combined_heuristic }
black_algorithm = RadioButtons(
    options=algorithms.keys(),
    value='Menschlicher Spieler',
    description='Schwarz:'
)
black_heuristic = RadioButtons(
    options=heuristics.keys(),
    value='Kombiniert'
)
black_mode = RadioButtons(
    options=modes.keys(),
    value='Zeitbegrenzte Vertiefung'
)
black_depth = IntSlider(value=5, min=1, max=10,
                        description='Suchtiefe:')
black_timelimit = IntSlider(value=30, min=1, max=120,
                            description='Zeitlimit:')
black_ints = VBox([black_depth, black_timelimit])
black_config = HBox([black_algorithm, black_mode,
                     black_heuristic, black_ints])

white_algorithm = RadioButtons(
    options=algorithms.keys(),
    value='ProbCut',
    description='Weiß:'
)
white_heuristic = RadioButtons(
    options=heuristics.keys(),
    value='Kombiniert'
)
white_mode = RadioButtons(
    options=modes.keys(),
    value='Zeitbegrenzte Vertiefung'
)
white_depth = IntSlider(value=5, min=1, max=10,
                        description='Suchtiefe:')
white_timelimit = IntSlider(value=30, min=1, max=120,
                            description='Zeitlimit:')
white_ints = VBox([white_depth, white_timelimit])
white_config = HBox([white_algorithm, white_mode,
                     white_heuristic, white_ints])
settings = ipywidgets.VBox([black_config, white_config])
\end{lstlisting}

Die Funktion \passthrough{\lstinline!configure\_settings!} dient der
Anzeige des Konfigurationsmenüs.

\begin{lstlisting}[language=Python]
def configure_settings():
    display(settings)
\end{lstlisting}

Die Funktion \passthrough{\lstinline!get\_settings!} liefert die über
das Konfigurationsmenü getätigte Konfiguration als Dictionary zurück.
Dies wird später im Frontend verwendet.

\begin{lstlisting}[language=Python]
def get_settings():
    return { BLACK: { 'heuristic': heuristics[black_heuristic.value],
                      'algorithm': algorithms[black_algorithm.value],
                      'depth': black_depth.value,
                      'timelimit': black_timelimit.value,
                      'mode': modes[black_mode.value] },
             WHITE: { 'heuristic': heuristics[white_heuristic.value],
                      'algorithm': algorithms[white_algorithm.value],
                      'depth': white_depth.value,
                      'timelimit': white_timelimit.value,
                      'mode': modes[white_mode.value] }}
\end{lstlisting}
