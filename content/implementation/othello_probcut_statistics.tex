\begin{lstlisting}[language=Python]
%%HTML
<style>
.container { width:100% }
</style>
\end{lstlisting}

\begin{lstlisting}[language=Python]
%run othello_game.ipynb
%run othello_ai.ipynb
\end{lstlisting}

\hypertarget{ermitteln-der-standardabweichung}{%
\paragraph{Ermitteln der
Standardabweichung}\label{ermitteln-der-standardabweichung}}

Im Folgenden werden zunächst einige Datenpunkte gesammelt, indem in
verschiedenen Spielzuständen jeweils eine tiefe, und eine flache Suche
durchgeführt wird. Die Spielzustände werden durch zufälliges Ziehen
erreicht.

\begin{lstlisting}[language=Python]
import csv
import os


def sample_probcut_values(num_games, shallow_depth, deep_depth):
    filename = f'probcut_dataset_{PROBCUT_SHALLOW_DEPTH}_{PROBCUT_DEEP_DEPTH}.csv'
    file_exists = os.path.isfile(filename)
    if file_exists:
        print('using existing dataset')
    else:
        with open(filename, 'w', newline='') as file:
            writer = csv.writer(file, delimiter=',')
            writer.writerow(('moves', 'shallow', 'deep'))
            for i in range(num_games):
                state = GameState()
                while not state.game_over:
                    ai_make_move(random_ai, state, 0, None)
                    shallow_value = alphabeta(
                        state, PROBCUT_SHALLOW_DEPTH, combined_heuristic, -math.inf, math.inf)
                    deep_value = alphabeta(
                        state, PROBCUT_DEEP_DEPTH, combined_heuristic, -math.inf, math.inf)
                    print(f'shallow: {shallow_value}, deep: {deep_value}')
                    writer.writerow(
                        (state.num_pieces, shallow_value, deep_value))
            file.close()
\end{lstlisting}

\begin{lstlisting}[language=Python]
sample_probcut_values(20, PROBCUT_SHALLOW_DEPTH, PROBCUT_DEEP_DEPTH)
\end{lstlisting}

Daten Laden

\begin{lstlisting}[language=Python]
import pandas
filename = 'probcut_dataset_2_4.csv'
df = pandas.read_csv(filename)
shallow = df['shallow']
deep = df['deep']
\end{lstlisting}

Berechnung der Standardabweichung nach Anzahl der Steine auf dem
Spielfeld

\begin{lstlisting}[language=Python]
import numpy as np
import sklearn.metrics as skl
import sklearn.linear_model as lm
import matplotlib.pyplot as plt
import seaborn as sns
import math

shallow_np = np.array(shallow)
deep_np = np.array(deep)
moves = np.array(df['moves'])

x = []
y = []

for i in range(5, 45):
    shallow_c = shallow[moves == i]
    deep_c = deep[moves == i]
    variance = np.var(np.stack([shallow_c, deep_c], axis=1))
    explained_variance = skl.explained_variance_score(shallow_c, deep_c)
    x.append(i)
    y.append(math.sqrt(variance))

model = lm.LinearRegression()
model.fit(np.array(x).reshape(-1, 1), np.array(y))

plt.figure(figsize=(15, 10))
sns.set(style='whitegrid')
plt.scatter(x, y)
plt.axvline(x=0.0, c='k')
plt.axhline(y=0.0, c='k')
plt.plot(x, x * model.coef_ + model.intercept_)
plt.xlabel('number of disks on the board')
plt.ylabel('standard deviation')
plt.title('Standard Deviation')
plt.show()
\end{lstlisting}

\begin{lstlisting}[language=Python]
print('x *', model.coef_[0], '+', model.intercept_)
\end{lstlisting}

\hypertarget{daten-visualisieren}{%
\paragraph{Daten Visualisieren}\label{daten-visualisieren}}

\begin{lstlisting}[language=Python]
import matplotlib.pyplot as plt
import sklearn.linear_model as lm
import seaborn as sns
model = lm.LinearRegression()
model.fit(shallow.values.reshape(len(shallow), 1), deep)
plt.figure(figsize=(15, 10))
sns.set(style='whitegrid')
plt.scatter(shallow, deep)
plt.axvline(x=0.0, c='k')
plt.axhline(y=0.0, c='k')
plt.plot(shallow, shallow * model.coef_ + model.intercept_)
plt.xlabel('shallow search')
plt.ylabel('deep search')
plt.title('Probcut Values')
plt.show()
\end{lstlisting}

Varianz, erklärte Varianz und Standardabweichung berechnen

\begin{lstlisting}[language=Python]
import numpy as np
import sklearn.metrics as skl
variance = np.var(np.stack([shallow, deep], axis=1))
print(f'variance: {variance}')
explained_variance = skl.explained_variance_score(shallow, deep)
print(f'standard distribution: {np.sqrt(variance)}')
print(f'explained variance: {explained_variance}')
\end{lstlisting}
