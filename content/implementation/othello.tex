\begin{lstlisting}[language=Python]
%%HTML
<style>
.container { width:100% }
</style>
\end{lstlisting}

\begin{lstlisting}[language=Python]
%run othello_ai.ipynb
%run othello_gui.ipynb
\end{lstlisting}

\begin{lstlisting}[language=Python]
import time
\end{lstlisting}

\hypertarget{applikation-starten}{%
\subsubsection{Applikation Starten}\label{applikation-starten}}

Im folgenden wird für beide Spieler die zu verwendende Künstliche
Intelligenz, sowie die jeweils angewandte Heuristik festgelegt. Ein Wert
von \passthrough{\lstinline!None!} bei der KI steht hierbei für einen
menschlichen Spieler. Die KIs und Heuristiken werden jeweils in einem
Dictionary gespeicher, sodass mit Spieler als Index auf die
entsprechende KI oder Heuristik zugegriffen werden kann.

\begin{lstlisting}[language=Python]
configure_settings()
\end{lstlisting}

Folgender Code dient zum Starten der interaktiven Applikation. Die
Funktion \passthrough{\lstinline!next\_move!} wird für jeden Spielzug
ausgeführt. Sie wird zu Beginn einmal aufgerufen. Wenn eine KI spielt,
wird die Funktion rekursiv für den nächsten Zug aufgerufen. Wenn der
Spieler menschlich ist, muss die Ausführung unterbrochen werden, da auf
das Aufrufen eines Callbacks durch einen Klick gewartet werden muss. Im
Callback wird auch die Funktion \passthrough{\lstinline!next\_move!} für
den nächsten Zug aufgerufen.

\begin{lstlisting}[language=Python]
state = GameState()
display_board(state)
settings = get_settings()

def next_move(state):
    # Check if/which AI is playing
    ai = settings[state.turn]['algorithm']
    if ai is not None:
        time.sleep(0.2)
        make_move = settings[state.turn]['mode']
        depth = settings[state.turn]['depth']
        heuristic = settings[state.turn]['heuristic']
        make_move(ai, state, depth, heuristic)
        update_output(state)
        if not state.game_over:
            next_move(state)
try:
    next_move(state)
except KeyboardInterrupt:
    pass
\end{lstlisting}
