\hypertarget{implementierung-einer-kuxfcnstlichen-intelligenz-fuxfcr-das-spiel-othello}{%
\section{Implementierung einer Künstlichen Intelligenz für das Spiel
Othello}\label{implementierung-einer-kuxfcnstlichen-intelligenz-fuxfcr-das-spiel-othello}}

\hypertarget{initiale-konfiguration}{%
\subsection{Initiale Konfiguration}\label{initiale-konfiguration}}

Importieren von Abhängigkeiten und Konfiguration

Die Implementierung der Spielelogik findet in einem separaten Jupyter
Notebook statt. Diese wird deshalb hier zunächst ausgeführt.

\begin{lstlisting}[language=Python]
%run othello_game.ipynb
\end{lstlisting}

\begin{lstlisting}[language=Python]
import math
import copy
import time
import numpy
import random
\end{lstlisting}

\hypertarget{heuristiken}{%
\subsection{Heuristiken}\label{heuristiken}}

Zum Abschätzen der Nützlichkeit eines Spielzustands wird eine Heuristik
benötigt. Im folgenden sind einige solcher Heuristiken implementiert. Da
Weiß der maximierende Spieler, und Schwarz der minimierende Spieler ist
repräsentiert ein höherer Wert der Heuristik einen für Weiß
vorteilhaften Zug, während ein niedriger Wert einen Vorteil für Schwarz
repräsentiert. Die Werte der Heuristik liegen zwischen -1 und 1, wobei
die Randwerte einen garantierten Sieg für den jeweiligen Spieler
darstellen. Der Wert 0 steht für einen für beide Spieler gleich guten
Spielzustand.

Da das Ziel des Spiels Othello ist, zum Ende des Spiels mehr Steine als
der Gegner auf dem Spielfeld zu haben, ist es naheliegend, die Differez
der Anzahlen von Steinen beider Spieler zur Abschätzung eines Zuges zu
verwenden. Genau das macht die Disk-Count-Heuristik indem sie die
Differenz der Steine, beider Spieler berechnet und den resultierenden
Wert zur Normalisierung durch die maximale Anzahl an Steinen teilt. Bei
genauerer Betrachtung ist es jedoch, gerade zu Beginn des Spiels nicht
immer vorteilhaft, den Vorsprung an Steinen zu maximieren.

\begin{lstlisting}[language=Python]
def disc_count_heuristic(state):
    return (count_disks(state, WHITE) - count_disks(state, BLACK)) / 64
\end{lstlisting}

Eine weitere Heuristik ist die Mobilität der Spieler. Diese gibt an wie
viele mögliche Züge ein Spieler gegenüber dessen Gegner machen kann. Die
Idee bei dieser Heuristik ist, dass ein Spieler dadurch seine Freiheit
maximiert, während die Freiheit des Gegners durch eine geringe Anzahl an
Zügen eingeschränkt wird. Die Mobilitäts-Heuristik gibt an wie viele
möglicher Züge mehr Weiß gegenüber Schwarz im Aktuellen Spielzustand
hat. Auch dieser Wert wird durch Division durch die Anzahl an Feldern
normalisiert, um die Grenzen von -1 und 1 einzuhalten.

\begin{lstlisting}[language=Python]
def mobility_heuristic(state):
    if state.turn == WHITE:
        return (len(state.possible_moves) - len(get_possible_moves(state, BLACK))) / 64
    else:
        return (len(get_possible_moves(state, WHITE)) - len(state.possible_moves)) / 64
\end{lstlisting}

Beim Spielen von Othello fällt auf, dass es bestimmte Felder gibt, deren
Belegung von Vorteil ist, sowie einige, deren Belegung eher nachteilhaft
ist. Diese Eigenschaft macht sich die Weight-Heuristik zu Nutze. Diese
weist jedem Feld einen Wert zu der angibt, wie Vorteilhaft der Besitz
dieses Feldes ist, bzw. wie Nachteilhaft die Belegung des Feldes durch
den Gegner ist. Diese Gewichte werden dann mit der aktuellen Belegung
des Spielfelds multipliziert und die Ergebnisse anschließend
aufsummiert. Der resultierende Wert schätzt dann den Nutzen der
aktuellen Position ein. Auch bei dieser Heuristik findet eine
Normalisierung statt.

Aufgrund der Symmetrie des Othello Spielfeldes ist es für die
Weight-Heuristik nicht nötig, für jedes Feld einzeln dessen Gewicht
anzugeben, stattdessen wird das Gewicht jeweils für charakteristische
felder, wie zum Beispiel die Ecken, angegeben. Die Funktion
\passthrough{\lstinline!gen\_weight\_matrix!} generiert dann die
Gewichte-Matrix für das gesamte Feld und führt auch die Normalisierung
durch.

\begin{lstlisting}[language=Python]
def gen_weight_matrix(default=0, corner=0, adj_corner=0, sup_corner=0, edge=0, dia_corner=0, support=0, sup_edge=0):
    raw_matrix = numpy.array([
        [corner,      adj_corner,  sup_corner,  edge,      edge,      sup_corner,  adj_corner,  corner],
        [adj_corner,  dia_corner,  default,     default,   default,   default,     dia_corner,  adj_corner],
        [sup_corner,  default,     support,     sup_edge,  sup_edge,  support,     default,     sup_corner],
        [edge,        default,     sup_edge,    default,   default,   sup_edge,    default,     edge],
        [edge,        default,     sup_edge,    default,   default,   sup_edge,    default,     edge],
        [sup_corner,  default,     support,     sup_edge,  sup_edge,  support,     default,     sup_corner],
        [adj_corner,  dia_corner,  default,     default,   default,   default,     dia_corner,  adj_corner],
        [corner,      adj_corner,  sup_corner,  edge,      edge,      sup_corner,  adj_corner,  corner]
    ])
    max_possible = numpy.sum(numpy.absolute(raw_matrix))
    return numpy.true_divide(raw_matrix, max_possible)
\end{lstlisting}

Die Funktion \passthrough{\lstinline!weight\_heuristic!} berechnet die
Weight-Heuristik allgemein für eine Gewichte-Matrix

\begin{lstlisting}[language=Python]
def weight_heuristic(state, weights):
    return numpy.sum(numpy.multiply(state.board, weights))
\end{lstlisting}

Die Gewichte-Matrix \passthrough{\lstinline!cowthello\_weights!} nutzt
die Gewichte aus dem Online-Othello Programm Cowthello. Cowthello ist
unter der URL \url{http://www.aurochs.org/games/cowthello/} verfügbar.

\begin{lstlisting}[language=Python]
cowthello_weights = gen_weight_matrix(default=1, corner=100, adj_corner=-25, sup_corner=25, edge=10, dia_corner=-50, support=50, sup_edge=5)
\end{lstlisting}

Die Funktion \passthrough{\lstinline!cowthello\_heuristic!} ist eine
Weight-Heuristik welche die Gewichte-Matrix von Cowthello verwendet.

\begin{lstlisting}[language=Python]
def cowthello_heuristic(state):
    return weight_heuristic(state, cowthello_weights)
\end{lstlisting}

Die oben implementierten Heuristiken bewerten jeweils nur ein Merkmal
der aktuellen Spielsitation. Durch eine Kombination mehrerer dieser
Heuristiken können mehrere Merkmale gleichzeitig betrachtet werden. Die
Kombination der Heuristiken kann in Abhängigkeit der aktuellen
Spielphase geschehen. Beispielsweise wird zu Ende des Spiels, die
Disc-Count-Heuristik wichtiger.

\begin{lstlisting}[language=Python]
def combined_heuristic(state):
    if(state.num_pieces >= 50):
        return disc_count_heuristic(state)
    mobility = mobility_heuristic(state)
    cowthello = cowthello_heuristic(state)
    return (mobility + cowthello) / 2
\end{lstlisting}

\hypertarget{implementierung-der-strategien}{%
\subsection{Implementierung der
Strategien}\label{implementierung-der-strategien}}

\hypertarget{zufuxe4llige-ki}{%
\subsubsection{Zufällige KI}\label{zufuxe4llige-ki}}

Diese KI wählt aus der Menge der Möglichen Züge einen zufälligen aus und
spielt diesen.

\begin{lstlisting}[language=Python]
def random_ai_make_move(state, heuristic):
    possible_moves = state.possible_moves
    random_move = random.choice(possible_moves)
    make_move(state, random_move[0], random_move[1])
\end{lstlisting}

\hypertarget{minimax-ki}{%
\subsubsection{Minimax KI}\label{minimax-ki}}

Diese KI verwendet den Minimax Algorithmus zur Bestimmung der
Nützlichkeit eines Zuges. Dies geschieht in der Funktion
\passthrough{\lstinline!minimax!}

\begin{lstlisting}[language=Python]
MINIMAX_DEPTH_LIMIT = 3
debug_mm_count= 0

def minimax(state, depth, heuristic):
    global debug_mm_count
    if(state.game_over):
        return get_winner(state)
    if(depth <= 0):
        return heuristic(state)
    
    if state.turn == WHITE:
        # maximizing
        utility = -math.inf
    else:
        # minimizing
        utility = math.inf
        
    for move in get_possible_moves(state, state.turn):
        debug_mm_count += 1
        tmp_state = copy.deepcopy(state)
        make_move(tmp_state, move[0], move[1])
        tmp_utility = minimax(tmp_state, depth - 1, heuristic)
        if state.turn == WHITE:
            # maximizing
            utility = max(utility, tmp_utility)
        else:
            # minimizing
            utility = min(utility, tmp_utility)          
    return utility
\end{lstlisting}

Für alle möglichen Züge wird die Nützlichkeit des resultierenden
Zustands mithilfe von \passthrough{\lstinline!minimax!} bestimmt und aus
allen Zügen mit einer, für den Spieler optimalen, Nützlichkeit ein
zufälliger Zug ausgewählt.

\begin{lstlisting}[language=Python]
def minimax_ai_make_move(state, heuristic):
    if(state.game_over):
        return
    scored_moves = [(minimax(make_move(copy.deepcopy(state), move[0], move[1]), MINIMAX_DEPTH_LIMIT-1, heuristic), move) for move in state.possible_moves]
    if state.turn == WHITE:
        # maximizing
        best_score, _ = max(scored_moves)
    else:
        # minimizing
        best_score, _ = min(scored_moves)
    best_move = random.choice([move for move in scored_moves if move[0] == best_score])[1]
    make_move(state, best_move[0], best_move[1])
\end{lstlisting}

\hypertarget{alpha-beta-ki}{%
\subsubsection{Alpha-Beta KI}\label{alpha-beta-ki}}

Diese KI verwended den Minimax Algorithmus mit der Optimierung
Alpha-Beta Pruning, um die Nützlichkeit eines Spielzustands zu
bestimmen.

Zum Merken vorheriger Ausführungen von wird das Dictionary
\passthrough{\lstinline!transposition\_table!} verwendet. Dies ist
gerade bei der Verwendung von Iterative Deepening für das Move-Ordering
vorteilhaft. Der Schlüssel des Dictionaries besteht aus dem Zustand des
Spielbretts, dem Spieler, der an der Reihe ist und der verwendeten
Heuristik.

\begin{lstlisting}[language=Python]
transposition_table = {}
\end{lstlisting}

Die Funktion \passthrough{\lstinline!alphabeta!} implementiert den
Minimax Algorithmus mit Alpha-Beta-Pruning

\begin{lstlisting}[language=Python]
ALPHABETA_DEPTH_LIMIT = 3
debug_ab_count= 0

def alphabeta(state, depth, alpha, beta, heuristic):
    global debug_ab_count
    if(state.game_over):
        return get_winner(state)
    if(depth <= 0):
        return heuristic(state)
    
    moves = get_possible_moves(state, state.turn)
    child_states = [make_move(copy.deepcopy(state), move[0], move[1]) for move in moves]
    estimated_utilities = [transposition_table[(child_state.board.tobytes(), child_state.turn, heuristic)]
                           if (child_state.board.tobytes(), child_state.turn, heuristic) in transposition_table
                           else heuristic(state)
                           for child_state in child_states]
    ordered_moves = [(estimated_utilities[i], child_states[i]) for i in range(len(moves))]
    ordered_moves.sort(reverse=(state.turn == WHITE))
    
    if state.turn == WHITE:
        # maximizing
        utility = -math.inf
    else:
        # minimizing
        utility = math.inf
        
    for (_,  tmp_state) in ordered_moves:
        debug_ab_count += 1
        tmp_utility = alphabeta(tmp_state, depth - 1, alpha, beta, heuristic)
        transposition_table[(tmp_state.board.tobytes(), tmp_state.turn, heuristic)] = tmp_utility
        
        if state.turn == WHITE:
            # maximizing
            utility = max(utility, tmp_utility)
            alpha = max(alpha, utility)
        else:
            # minimizing
            utility = min(utility, tmp_utility)
            beta = min(beta, utility)
        if(alpha >= beta):
            break # alphabeta pruning
    return utility
\end{lstlisting}

Wie bei der Minimax KI wird auch bei der Alpha-Beta KI für alle
möglichen Züge wird die Nützlichkeit des resultierenden Zustands
bestimmt, hier mithilfe von von \passthrough{\lstinline!alphabeta!}, und
aus allen Zügen mit einer, für den Spieler optimalen, Nützlichkeit ein
zufälliger Zug ausgewählt.

\begin{lstlisting}[language=Python]
def alphabeta_ai_make_move(state, heuristic):
    if(state.game_over):
        return
    scored_moves = [(alphabeta(make_move(copy.deepcopy(state), move[0], move[1]), ALPHABETA_DEPTH_LIMIT-1, -math.inf, math.inf, heuristic), move) for move in state.possible_moves]
    if state.turn == WHITE:
        # maximizing
        best_score, _ = max(scored_moves)
    else:
        # minimizing
        best_score, _ = min(scored_moves)
    best_move = random.choice([move for move in scored_moves if move[0] == best_score])[1]
    make_move(state, best_move[0], best_move[1])
\end{lstlisting}

Beim Iterative Deepening wird der Alphabeta-Algorithmus nacheinander mit
steigender Tiefe ausgeführt. Die Werte aus der vorherigen Ausführung
werden dann für das Move-Ordering verwendet. Dadurch das bessere
Move-Ordering können beim Alpha-Beta-Pruning mehr Zweige abgeschnitten
werden. Dadurch kann trotz der mehrfachen Ausführung eine höhere
Performanz bei gleichem Ergebnis erreicht werden. Ein weiterer Vorteil
ist, dass das Iterative Deepening jederzeit abgebrochen werden kann.
Dadurch kann in einer vorgegebenen Zeit, die in dieser Zeit maximal
mögliche Suchtiefe erreicht werden.

\begin{lstlisting}[language=Python]
def alphabeta_id_make_move(state, heuristic):
    best_move = None
    depth = 1
    while depth <= ALPHABETA_DEPTH_LIMIT:
        scored_moves = [(alphabeta(make_move(copy.deepcopy(state), move[0], move[1]), depth-1, -math.inf, math.inf, heuristic), move) for move in state.possible_moves]
        if state.turn == WHITE:
            # maximizing
            best_score, _ = max(scored_moves)
        else:
            # minimizing
            best_score, _ = min(scored_moves)
        best_move = random.choice([move for move in scored_moves if move[0] == best_score])[1]
        depth += 1
    make_move(state, best_move[0], best_move[1])
\end{lstlisting}

Beim Iterative Deepening ist jederzeit der beste Zug der letzten
Suchtiefe bekannt. Daher kann der Algorithmus verwendet werden, um in
einer vorgegebenen Zeit mit größtmöglicher Suchtiefe zu suchen. Dazu
wird die Dauer des nächsten Zugs anhand der Dauer des letzten Zugs
geschätzt. Somit kann es vorkommen, dass die Berechnung etwas länger
oder kürzer als die gegebene Zeit dauert. Dafür wird das Ergebnis aller
Berechnungen verwendet und die letzte Suche muss nicht abgebrochen
werden.

\begin{lstlisting}[language=Python]
MAX_SECS_PER_MOVE = 30

def alphabeta_id_make_move_timelimited(state, heuristic):
    best_move = None
    depth = 1
    start = time.time()
    last_time = 0
    while MAX_SECS_PER_MOVE - (time.time() - start) >= 4 * last_time:
        last_time_start = time.time()
        scored_moves = [(alphabeta(make_move(copy.deepcopy(state), move[0], move[1]), depth-1, -math.inf, math.inf, heuristic), move) for move in state.possible_moves]
        if state.turn == WHITE:
            # maximizing
            best_score, _ = max(scored_moves)
        else:
            # minimizing
            best_score, _ = min(scored_moves)
        best_move = random.choice([move for move in scored_moves if move[0] == best_score])[1]
        last_time = time.time() - last_time_start
        depth += 1
    print("Reached depth: ", depth-1)
    make_move(state, best_move[0], best_move[1])


\hypertarget{probcut-ki}{%
\subsubsection{ProbCut KI}\label{probcut-ki}}

An dieser Stelle beginnt die Implementierung der Kuenstlichen Intelligenz
mittels des Minimax Algorithmus und ProbCut

\begin{lstlisting}[language=Python]
PERCENTILE = 1.5
PROBCUT_SHALLOW_DEPTH = 3
PROBCUT_DEPTH = 6
PROBCUT_SIGMA = 0.75

def probcut(state, depth, alpha, beta, heuristic):
    global debug_ab_count
    global alphabeta_best_move
    if(state.game_over):
        return get_winner(state)
    if(depth <= 0):
        return heuristic(state)
    
    if depth == PROBCUT_DEPTH:
        # v >= beta with prob. of at least p? yes => cutoff
        bound = PERCENTILE * PROBCUT_SIGMA + beta
        if probcut(state, PROBCUT_SHALLOW_DEPTH, bound-1, bound, heuristic) >= bound:
            return beta

        # v <= alpha with prob. of at least p? yes => cutoff
        bound = -PERCENTILE * PROBCUT_SIGMA + alpha
        if probcut(state, PROBCUT_SHALLOW_DEPTH, bound, bound+1, heuristic) <= bound:
            return alpha
    
    moves = get_possible_moves(state, state.turn)
    child_states = [make_move(copy.deepcopy(state), move[0], move[1]) for move in moves]
    estimated_utilities = [transposition_table[(child_state.board.tobytes(), child_state.turn, heuristic)]
                           if (child_state.board.tobytes(), child_state.turn, heuristic) in transposition_table
                           else heuristic(state)
                           for child_state in child_states]
    ordered_moves = [(estimated_utilities[i], moves[i], child_states[i]) for i in range(len(moves))]
    ordered_moves.sort(reverse=(state.turn == WHITE))
    
    if state.turn == WHITE:
        # maximizing
        utility = -math.inf
    else:
        # minimizing
        utility = math.inf
        
    for (_, move, tmp_state) in ordered_moves:
        debug_ab_count += 1
        tmp_utility = alphabeta(tmp_state, depth - 1, alpha, beta, heuristic)
        transposition_table[(tmp_state.board.tobytes(), tmp_state.turn, heuristic)] = tmp_utility
        
        if state.turn == WHITE:
            # maximizing
            if(tmp_utility > utility):
                utility = tmp_utility
                if(depth == ALPHABETA_DEPTH_LIMIT):
                    alphabeta_best_move = move
            alpha = max(alpha, utility)
        else:
            # minimizing
            if(tmp_utility < utility):
                utility = tmp_utility
                if(depth == ALPHABETA_DEPTH_LIMIT):
                    alphabeta_best_move = move
            beta = min(beta, utility)
        if(alpha >= beta):
            break # alphabeta pruning
    return utility
\end{lstlisting}

\begin{lstlisting}[language=Python]
def probcut_ai_make_move(state, heuristic):
    if(state.game_over):
        return
    scored_moves = [(probcut(make_move(copy.deepcopy(state), move[0], move[1]), PROBCUT_DEPTH-1, -1, 1, heuristic), move) for move in state.possible_moves]
    if state.turn == WHITE:
        # maximizing
        _, best_move = max(scored_moves)
    else:
        # minimizing
        _, best_move = min(scored_moves)
    make_move(state, best_move[0], best_move[1])
\end{lstlisting}

\hypertarget{applikation-starten}{%
\subsubsection{Applikation Starten}\label{applikation-starten}}

Im folgenden wird für beide Spieler die zu verwendende Künstliche
Intelligenz, sowie die jeweils angewandte Heuristik festgelegt. Ein Wert
von \passthrough{\lstinline!None!} bei der KI steht hierbei für einen
nenschlichen Spieler

\begin{lstlisting}[language=Python]
# Settings
BLACK_PLAYER_AI = random_ai_make_move
WHITE_PLAYER_AI = alphabeta_id_make_move

BLACK_PLAYER_HEURISTIC = combined_heuristic
WHITE_PLAYER_HEURISTIC = combined_heuristic

PLAYER_AI = {BLACK: BLACK_PLAYER_AI, WHITE: WHITE_PLAYER_AI}
PLAYER_HEURISTIC = {BLACK: BLACK_PLAYER_HEURISTIC, WHITE: WHITE_PLAYER_HEURISTIC}
\end{lstlisting}

Zum Starten der interaktiven Applikation wird der folgende Code
ausgeführt

\begin{lstlisting}[language=Python]
state = GameState()
display_board(state)

def next_move(state):
    time.sleep(0.2)
    # Check if/which AI is playing
    strat = BLACK_PLAYER_AI if state.turn == BLACK else WHITE_PLAYER_AI
    if strat is not None:
        strat(state, PLAYER_HEURISTIC[state.turn])
        update_output(state)
        if not state.game_over:
            next_move(state)

try:
    next_move(state)
except KeyboardInterrupt:
    pass
\end{lstlisting}

\begin{lstlisting}
MultiCanvas(height=560, width=560)



Label(value='Black Player : 2 White Player : 2')



Label(value='Blacks Move')



Output()
\end{lstlisting}

\hypertarget{testing-code}{%
\subsubsection{Testing code}\label{testing-code}}

Der folgende Code dient zum Test, sowie zum debugging der oben
Implementierten Funktionen

\begin{lstlisting}[language=Python]
test_board = GameState()
\end{lstlisting}

\begin{lstlisting}[language=Python]
alphabeta_ai_make_move(test_board, combined_heuristic)
test_board.board
\end{lstlisting}

\begin{lstlisting}[language=Python]
import cProfile

cProfile.run('alphabeta_ai_make_move(test_board, combined_heuristic)')
test_board.board
\end{lstlisting}

\begin{lstlisting}[language=Python]
def debug_num_visited_states(state):
    global debug_ab_count
    global debug_mm_count
    debug_mm_count= 0
    start = time.time()
    minimax_ai_make_move(copy.deepcopy(state), combined_heuristic)
    secs = time.time() - start
    print("Minimax takes ", secs, " seconds and checks ", debug_mm_count, "substates")
    debug_ab_count= 0
    transposition_table = {}
    start = time.time()
    alphabeta_ai_make_move(copy.deepcopy(state), combined_heuristic)
    secs = time.time() - start
    print("AlphaBeta takes ", secs, " seconds and checks ", debug_ab_count, "substates")
    debug_ab_count= 0
    transposition_table = {}
    start = time.time()
    alphabeta_id_make_move(copy.deepcopy(state), combined_heuristic)
    secs = time.time() - start
    print("AlphaBeta with iterative deepening takes ", secs, " seconds and checks ", debug_ab_count, "substates")
\end{lstlisting}

\begin{lstlisting}[language=Python]
debug_num_visited_states(test_board)
\end{lstlisting}

\begin{lstlisting}[language=Python]
def get_statistics(num, black_ai, black_h, white_ai, white_h):
    status = ipywidgets.widgets.Label()
    display(status)
    result = []
    wins = [0, 0, 0]
    status.value = f'0 / {num} games played, b/d/w: {wins[0]}/{wins[1]}/{wins[2]}'
    try:
        for i in range(num):
            (b, w) = play_game(black_ai, black_h, white_ai, white_h)
            result.append((b, w))
            if b > w:
                wins[0] += 1
            elif w == b:
                wins[1] += 1
            else:
                wins[2] += 1
            status.value = f'{i+1} / {num} games played, b/d/w: {wins[0]}/{wins[1]}/{wins[2]}'
    except KeyboardInterrupt:
        status.value = f'Interrupted: {i} / {num} games played, b/d/w: {wins[0]}/{wins[1]}/{wins[2]}'
    print_statistics(result)

def play_game(black_ai, black_h, white_ai, white_h):
    state = GameState()
    next_move_blind(state, black_ai, white_ai, {BLACK: black_h, WHITE: white_h})
    return count_disks(state, BLACK), count_disks(state, WHITE)


def next_move_blind(state, black_ai, white_ai, heuristics):
    # Check if/which AI is playing
    strat = black_ai if state.turn == BLACK else white_ai
    strat(state, heuristics[state.turn])
    if not state.game_over:
        next_move_blind(state, black_ai, white_ai, heuristics)

def print_statistics(results):
    print(results)
\end{lstlisting}

\begin{lstlisting}[language=Python]
get_statistics(5, random_ai_make_move, cowthello_heuristic, random_ai_make_move, cowthello_heuristic)
\end{lstlisting}

\hypertarget{legacy-code}{%
\section{Legacy code}\label{legacy-code}}

def alphabeta\_max(state, alpha, beta, depth): global
alphabeta\_best\_move if(state.game\_over): return
terminal\_utility(state) if(depth \textgreater=
ALPHABETA\_DEPTH\_LIMIT): return heuristic\_utility(state) max\_utility
= -math.inf for move in state.get\_possible\_moves(): tmp\_state =
copy.deepcopy(state) tmp\_state.move(move{[}0{]}, move{[}1{]})
tmp\_utility = alphabeta\_min(tmp\_state, alpha, beta, depth + 1)
if(tmp\_utility \textgreater{} max\_utility): max\_utility =
tmp\_utility if(depth == 0): alphabeta\_best\_move = move
if(max\_utility \textgreater= beta): return max\_utility alpha =
max(alpha, max\_utility) return max\_utility

def alphabeta\_min(state, alpha, beta, depth): global
alphabeta\_best\_move if(state.game\_over): return
-terminal\_utility(state) if(depth \textgreater=
ALPHABETA\_DEPTH\_LIMIT): return -heuristic\_utility(state) min\_utility
= math.inf for move in state.get\_possible\_moves(): tmp\_state =
copy.deepcopy(state) tmp\_state.move(move{[}0{]}, move{[}1{]})
tmp\_utility = alphabeta\_max(tmp\_state, alpha, beta, depth + 1)
if(tmp\_utility \textless{} min\_utility): min\_utility = tmp\_utility
if(depth == 0): alphabeta\_best\_move = move if(min\_utility \textless=
alpha): return min\_utility beta = min(beta, min\_utility) return
min\_utility

\begin{lstlisting}[language=Python]
\end{lstlisting}

ALPHABETA\_DEPTH\_LIMIT = 4

\#TODO: What if one player has to play twice in a row? def
alphabeta\_negamax(state, depth, alpha, beta): global
alphabeta\_best\_move if(state.game\_over): return
terminal\_utility(state) * state.turn if(depth == 0): if(state.turn ==
WHITE): return WHITE\_PLAYER\_HEURISTIC(state) else: return
BLACK\_PLAYER\_STRAT\_HEURISTIC(state) * state.turn utility = -math.inf
for move in get\_possible\_moves(state, state.turn): tmp\_state =
copy.deepcopy(state) make\_move(tmp\_state, move{[}0{]}, move{[}1{]})
tmp\_utility = -alphabeta\_negamax(tmp\_state, depth - 1, -beta, -alpha)
if(tmp\_utility \textgreater{} utility): utility = tmp\_utility if(depth
== ALPHABETA\_DEPTH\_LIMIT): alphabeta\_best\_move = move if(utility
\textgreater= beta): return utility alpha = max(alpha, utility) return
utility

def alphabeta\_ai\_make\_move(state): alphabeta\_negamax(state,
ALPHABETA\_DEPTH\_LIMIT, -math.inf, math.inf) make\_move(state,
alphabeta\_best\_move{[}0{]}, alphabeta\_best\_move{[}1{]})
