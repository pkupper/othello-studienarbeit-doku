\begin{lstlisting}[language=Python]
%%HTML
<style>
.container { width:100% }
</style>
\end{lstlisting}

\hypertarget{implementierung-der-spiellogik}{%
\section{Implementierung der
Spiellogik}\label{implementierung-der-spiellogik}}

Im Folgenden ist die Spielelogik des Spiels Othello implementiert. Dazu
gehört die Implementation aller im \autoref{sec:spieltheorie} genannten
Aspekte, wie zum Beispiel die erzeugung eines Startzustands, sowie die
die Bestimmung und Durchführung von Spielzügen ausgehend von einem
Spielzustand. Ausgangspunkt für diese Implementierung ist das Python Gui
für Othello von Kevan Nguyen, welches unter
\url{https://github.com/kevannguyen/Othello} verfügbar ist.

\hypertarget{importieren-der-externen-abhuxe4ngigkeiten}{%
\subsection{Importieren der externen
Abhängigkeiten}\label{importieren-der-externen-abhuxe4ngigkeiten}}

Die Implementation stützt sich für bessere Performanz auf die
Python-Bibliothek \passthrough{\lstinline!numpy!}, welche unter anderem
homogene Felder und Matrizen implementiert. Eine solche Matrix wird als
interne Repräsentation des Othello Spielfeldes genutzt. Insbesondere
Operationen, die auf einen größeren Teil des Spielfeldes zugreifen
müssen werden, können dadurch beschleunigt werden.

Für das Kopieren der Spielzustände wird das Modul
\passthrough{\lstinline!copy!} aus der Python Standardbibliothek
verwendet.

\begin{lstlisting}[language=Python]
import numpy as np
import copy
\end{lstlisting}

\hypertarget{globale-konstanten}{%
\subsection{Globale Konstanten}\label{globale-konstanten}}

Im Folgenden Abschnitt werden zunächst einige Konstanten definiert,
welche in der späteren Implementation häufig genutzt werden.

Die Konstante \passthrough{\lstinline!BOARD\_SIZE!} gibt die Anzahl an
Zeilen und Spalten des quadratischen Othello Spielfelds an.
\passthrough{\lstinline!BOARD\_SIZE!} wird beispielsweise zur Iteration
über Zeilen und Spalten des Spielfeldes genutzt, sowie zur Überprüfung,
ob gegebene Koordinaten innerhalb des Spielfeldes liegen.

Die Konstanten \passthrough{\lstinline!BLACK!},
\passthrough{\lstinline!WHITE!} und \passthrough{\lstinline!NONE!}
werden auf die Zahlenwerte -1 , 1 und 0 abgebildet und werden für
mehrere Zwecke genutzt:

\begin{enumerate}
\def\labelenumi{\arabic{enumi}.}
\tightlist
\item
  Repräsentation des Spielfeldes: Das Othello Spielbrett wird als
  \(8\times 8\) Matrix von Ganzzahlen definiert, welche jeweils einen
  der drei Werte annehmen können. Hierbei stehen die Werte
  \passthrough{\lstinline!BLACK!} und \passthrough{\lstinline!WHITE!}
  jeweils für einen Stein des jeweiligen Spielers, während
  \passthrough{\lstinline!NONE!} ein leeres Feld repräsentiert.
\item
  Repräsentation der Spieler: \passthrough{\lstinline!BLACK!} und
  \passthrough{\lstinline!WHITE!} werden zur Repäsentation eines
  Spielers genutzt. Beispielsweise enthält der Spielzustand eine
  Variable \passthrough{\lstinline!turn!} die angibt, welcher Spieler am
  Zug ist.
\item
  Berechnung der Heuristiken: Die Werte \passthrough{\lstinline!BLACK!},
  \passthrough{\lstinline!WHITE!} und \passthrough{\lstinline!NONE!}
  sind so gewählt, dass sie sich für die Berechnung der Heuristiken
  eignen. Der in der Künstlichen Intelligenz maximierende Spieler hat
  den positiven Wert 1, während der minimierende Spieler durch den
  negativen Wert -1 repräsentiert wird. Kein Spieler, wird durch den
  Wert 0 dargestellt.
\end{enumerate}

\begin{lstlisting}[language=Python]
BOARD_SIZE = 8

BLACK = -1  # MINIMIZING PLAYER
WHITE = 1  # MAXIMIZING PLAYER
NONE = 0 # NO PLAYER
\end{lstlisting}

\hypertarget{game-state}{%
\subsection{Game State}\label{game-state}}

Die Klasse \passthrough{\lstinline!GameState!} repäsentiert einen
Spielzustand von Othello. Dieser wird durch die im Folgenden genannten
Attribute beschrieben

\begin{itemize}
\tightlist
\item
  Das Spielfeld \passthrough{\lstinline!board!}, welches durch eine
  zweidimensionale Numpy Matrix repräsentiert wird, bei der jede Zelle
  die die Werte \passthrough{\lstinline!BLACK!},
  \passthrough{\lstinline!WHITE!} und \passthrough{\lstinline!NONE!}
  annehmen kann.
\item
  Den Spieler \passthrough{\lstinline!turn!}, der im Spielzustand am Zug
  ist. Zusätzlich enthält der Spielzustand weitere Informationen welche
  zur Verbesserung der Performanz genutzt werden.
\item
  Die im aktuellen Spielzustand möglichen Züge, werden als Paare von
  Koordinaten in der Variable \passthrough{\lstinline!possible\_moves!}
  gespeichert werden. Ein Koordinatenpaar steht hierbei für das Setzen
  eines Spielsteins auf die entsprechende Stelle auf dem Spielfeld unter
  Anwendung der Othello Regeln. Diese Variable ist Teil von
  \passthrough{\lstinline!GameState!} damit pro Spielzustand, die
  möglichen Züge nicht mehrfach berechnet werden müssen.
\item
  Die Menge der freien Felder, die horizontal, vertikal oder diagonal an
  einen Stein angrenzen werden in der Variable
  \passthrough{\lstinline!frontier!} gespeichert. Beim Ermitteln der
  möglichen Züge kann dadurch die Performanz wesentlich gesteigert
  werden, da nur diese Menge und nicht das gesamte Spielfeld überprüft
  werden muss.
\item
  Die Anzahl an Spielsteinen auf dem Spielfeld wird in der Variable
  \passthrough{\lstinline!num\_pieces!} gehalten.
\item
  Ob der Spielzustand ein Endzustand ist, wird in der Variable
  \passthrough{\lstinline!game\_over!} gespeichert.
\item
  Die Koordinaten des letzten Spielzugs werden zur späteren
  Visualisierung in der Grafischen Benutzeroberfläche in der Variable
  \passthrough{\lstinline!last\_move!} gespeichert.
\end{itemize}

Die zur Performance-Verbesserung genutzen Variablen werden im Laufe des
Spielverlaufs immer aktuell gehalten.

In dem Konstruktor \passthrough{\lstinline!\_\_init\_\_!} der Klasse
\passthrough{\lstinline!GameState!} wird ein neuer Spielzustand
entsprechend den Othello Spielregeln instanziert, indem alle Variablen
entsprechen initialisiert werden.

Die Funktion \passthrough{\lstinline!\_\_lt\_\_!} ist implementiert,
damit auf \passthrough{\lstinline!GameState!}-Objekten der
Vergleichsoperator angewendet werden kann. Das ist nötig, damit in der
Künstlichen Intelligenz Tupel sortiert werden können, die beispielsweise
aus einer Priorität und einen \passthrough{\lstinline!GameState!}
bestehen. Da in diesem Fall nur die Sortierung nach der Priorität
wichtig ist, ist die implementation von
\passthrough{\lstinline!\_\_lt\_\_!} irrelevant. Daher wird, der
Einfachkeit wegen, immer True zurückgegeben.

\begin{lstlisting}[language=Python]
class GameState:
    def __init__(self):
        self.board = np.zeros((BOARD_SIZE, BOARD_SIZE), dtype=np.int8)
        self.board[3, 3] = WHITE
        self.board[3, 4] = BLACK
        self.board[4, 3] = BLACK
        self.board[4, 4] = WHITE
        self.turn = BLACK
        self.possible_moves = [(2, 3), (3, 2), (4, 5), (5, 4)]
        self.frontier = {(2, 2), (2, 3), (2, 4), (2, 5),
                         (3, 2), (3, 5), (4, 2), (4, 5),
                         (5, 2), (5, 3), (5, 4), (5, 5)}
        self.num_pieces = 4
        self.game_over = False
        self.last_move = None

    def __lt__(self, other):
        return True
\end{lstlisting}

Die Liste \passthrough{\lstinline!directions!} enthält alle
horizontalen, vertikalen und diagonalen Richtungen auf dem Spielfeld als
Zwei-Tupel. Die beiden Zahlen stellen hierbei jeweils den Versatz in
Reihen- und Spaltenrichtung dar. Diese Liste wird später in mehreren
Funktionen genutzt.

\begin{lstlisting}[language=Python]
directions = [(-1,-1),(0,-1),(1,-1),(-1,0),(1,0),(-1,1),(0,1),(1,1)]
\end{lstlisting}

Die Exception \passthrough{\lstinline!InvalidMoveException!}, wird
später in der Funktion \passthrough{\lstinline!make\_move!} geworfen,
wenn ein ungültigen Spielzug gefordert wird. Dies dient der
Fehlerbehandlung.

\begin{lstlisting}[language=Python]
class InvalidMoveException(Exception):
    pass
\end{lstlisting}

Die Funktion \passthrough{\lstinline!can\_flip\_in\_dir!} überprüft für
ein Spielfeld \passthrough{\lstinline!board!} und den Spieler
\passthrough{\lstinline!player!}, ob beim Setzen eines Steins auf die
Position \passthrough{\lstinline!(row, col)!} in die durch
\passthrough{\lstinline!(rowdelta, coldelta)!} gegebene Richtung Steine
umgedreht werden können. Die Variable \passthrough{\lstinline!board!}
enthält das Spielfeld als Python-Liste, da einzelne Zugriffe so deutlich
schneller sind, als auf ein numpy array.

\begin{lstlisting}[language=Python]
def can_flip_in_dir(board, row, col, rowdelta, coldelta, player):
    current_row = row + rowdelta
    current_col = col + coldelta
    if not (0 <= current_row < 8 and 0 <= current_col < 8):
        return False
    if not board[current_row][current_col] == -player:
        return False
    current_row += rowdelta
    current_col += coldelta
    
    while True:
        if not (0 <= current_row < 8 and 0 <= current_col < 8):
            return False
        if board[current_row][current_col] == NONE:
            return False           
        if board[current_row][current_col] == player:
            return True
    
        current_row += rowdelta
        current_col += coldelta
\end{lstlisting}

Die Funktion \passthrough{\lstinline!is\_move\_valid!} überprüft für ein
gegebenes Spielfeld \passthrough{\lstinline!board!}, ob ein Zug auf die
angegebenen Koordinaten \passthrough{\lstinline!row!} und
\passthrough{\lstinline!col!} für den Spieler
\passthrough{\lstinline!player!} möglich ist. Das Ergebnis wird als
Wahrheitswert zurückgegeben. \passthrough{\lstinline!board!} ist hier
ebenfalls eine Python-Liste, da der Zugriff auf Elemente schneller ist,
als in einem numpy array.

\begin{lstlisting}[language=Python]
def is_move_valid(board, row, col, player):
    for rowdelta, coldelta in directions:
        if can_flip_in_dir(board, row, col, rowdelta, coldelta, player):
            return True
    return False
\end{lstlisting}

\passthrough{\lstinline!get\_utility!} bestimmt für einen Endzustand den
Gewinner des Spiels. Gewinnt Weiß, so wird der Wert 1 zurückgegeben.
Gewinnt Schwarz, wird der Wert -1 zurückgegeben. Bei einem Unentschieden
wird der Wert 0 zurückgegeben.

\begin{lstlisting}[language=Python]
def get_utility(state):
    black_disks = count_disks(state, BLACK)
    white_disks = count_disks(state, WHITE)
    if black_disks > white_disks:
        return BLACK
    if white_disks > black_disks:
        return WHITE
    else:
        return NONE
\end{lstlisting}

Die Funktion \passthrough{\lstinline!get\_possible\_moves!} bestimmt für
einen Spielzustand \passthrough{\lstinline!state!} und den Spieler
\passthrough{\lstinline!player!} die möglichen Züge, die der Spieler
machen kann. Die Züge werden als Liste von Koordinaten zurückgegeben.

\begin{lstlisting}[language=Python]
def get_possible_moves(state, player):
    board = state.board.tolist()
    possible_moves = []
    for (row, col) in state.frontier:
        if is_move_valid(board, row, col, player):
            possible_moves.append((row, col))
    return possible_moves
\end{lstlisting}

Diese Funktion dreht im Spielzustand \passthrough{\lstinline!state!},
ausgehend von dem durch \passthrough{\lstinline!row!} und
\passthrough{\lstinline!col!} gegebenen Feld, die für den Spieler
\passthrough{\lstinline!player!} generischen Steine in die durch
\passthrough{\lstinline!rowdelta!} und
\passthrough{\lstinline!coldelta!} gegebene Richtung um.

\begin{lstlisting}[language=Python]
def flip_in_dir(state, row, col, rowdelta, coldelta, player):
    current_row = row + rowdelta
    current_col = col + coldelta
    
    while state.board[current_row, current_col] == -player:
        state.board[(current_row, current_col)] = player
        current_row += rowdelta
        current_col += coldelta
\end{lstlisting}

\passthrough{\lstinline!update\_frontier!} wird nach jedem Zug
aufgerufen um die Menge Frontier zu aktualisieren. Die durch
\passthrough{\lstinline!row!} und \passthrough{\lstinline!col!} Gegebene
Koordinate wird entfernt, während die Koordinaten von leeren umliegenden
Feldern hinzugefügt werden.

\begin{lstlisting}[language=Python]
def update_frontier(state, row, col):
    for current_row in range(row-1, row+2):
        if not 0 <= current_row < 8:
            continue
        for current_col in range(col-1, col+2):
            if not 0 <= current_col < 8:
                continue
            if state.board[current_row, current_col] == NONE:
                state.frontier.add((current_row, current_col))
    state.frontier.remove((row, col))
\end{lstlisting}

Die Funktion \passthrough{\lstinline!count\_disks!} zählt die Steine,
die der Spieler \passthrough{\lstinline!player!} im Spielzustand
\passthrough{\lstinline!state!} auf dem Spielfeld hat.

\begin{lstlisting}[language=Python]
def count_disks(state, player):
    return np.count_nonzero(state.board == player)
\end{lstlisting}

\passthrough{\lstinline!get\_player\_string!} konvertiert die ID des
Spielers \passthrough{\lstinline!player!} in dessen Name. Ist
\passthrough{\lstinline!player == NONE!} so wird `Nobody' zurückgegeben.

\begin{lstlisting}[language=Python]
def get_player_string(player):
    return {BLACK: 'Black', WHITE: 'White', NONE: 'Nobody'}[player]
\end{lstlisting}

Die Funktion \passthrough{\lstinline!make\_move!} versucht auf einem
Spielzustand \passthrough{\lstinline!state!} einen Spielzug entsprechend
den Othello Regeln auszuführen. Der auszuführende Zug wird hierbei durch
den Parameter \passthrough{\lstinline!pos!} bestimmt, welcher die
Spielfeldkoordinaten des zu setzenden Steins als Zwei-Tupel angibt.

Zunächst wird überprüft, ob die Koordinaten
\passthrough{\lstinline!pos!} bereits in der Variable
\passthrough{\lstinline!frontier!} enthalten ist. Ist dies nicht der
Fall, so kann die Funktion mit einer
\passthrough{\lstinline!InvalidMoveException!} abgebrochen werden, da
ein Spielstein nur auf ein leeres Feld gesetzt werden kann, welches an
einen Spielstein angrenzt. Hierbei handelt es sich um eine Maßnahme zur
Performanceoptimierung.

Ist \passthrough{\lstinline!pos!} in \passthrough{\lstinline!frontier!},
so wird der Spielzustand kopiert, um den ursprünglich übergebenen
Spielzustand nicht zu verändern.

Anschließend werden alle gegnerischen Steine, welche vom neu gesetzten
Stein eingeschlossen werden umgedreht. Wenn mindestens ein Stein
umgedreht wurde, wird \passthrough{\lstinline!disks\_flipped!} auf
\passthrough{\lstinline!True!}gesetzt. Wenn das nicht der Fall war,
handelt es sich nicht um einen gültigen Zug und es wird eine
\passthrough{\lstinline!InvalidMoveException!} geworfen. Andernfalls
wird der neue Stein gesetzt und im resultierenden Zustand die Variablen
\passthrough{\lstinline!frontier!}, \passthrough{\lstinline!turn!},
\passthrough{\lstinline!game\_over!} und
\passthrough{\lstinline!possible\_moves!} aktualisiert.

Der Rückgabewert der Funktion ist der neue Spielzustand.

\begin{lstlisting}[language=Python]
def make_move(state, pos):
    if pos not in state.frontier:
        print(pos, "not in Frontier")
        raise InvalidMoveException
    
    state = copy.deepcopy(state)
    disks_flipped = False
    (row, col) = pos
    board = state.board.tolist()
    for (row_dir, col_dir) in directions:
        if can_flip_in_dir(board, row, col, row_dir, col_dir, state.turn):
            disks_flipped = True
            flip_in_dir(state, row, col, row_dir, col_dir, state.turn)

    if disks_flipped:
        state.num_pieces += 1
        state.board[pos] = state.turn
        state.last_move = pos
        update_frontier(state, row, col)
        state.turn = -state.turn
        state.possible_moves = get_possible_moves(state, state.turn)
        if len(state.possible_moves) == 0:
            state.turn = -state.turn
            state.possible_moves = get_possible_moves(state, state.turn)
            if len(state.possible_moves) == 0:
                state.game_over = True
                return state
    else:
        raise InvalidMoveException()
    return state
\end{lstlisting}

\passthrough{\lstinline!make\_state!} erzeugt einen Spielzustand aus dem
Spielfeld \passthrough{\lstinline!board!} und dem zu ziehenden Spieler
\passthrough{\lstinline!turn!}. Diese Funktion wird zu Testzwecken
genutzt.

\begin{lstlisting}[language=Python]
def make_state(board, turn):
    state = GameState()
    state.board = board
    state.turn = turn
    state.frontier = set()
    for row in range(8):
        for col in range(8):
            for current_row in range(row-1, row+2):
                if not 0 <= current_row < 8:
                    continue
                for current_col in range(col-1, col+2):
                    if not 0 <= current_col < 8:
                        continue
                    if state.board[current_row, current_col] == NONE:
                        state.frontier.add((current_row, current_col))
            
    state.possible_moves = get_possible_moves(state, turn)
    state.game_over = False
    if len(state.possible_moves) == 0:
        if len(get_possible_moves(state, -turn)) == 0:
            state.game_over = True
    state.num_pieces = count_disks(state, WHITE) + count_disks(state, BLACK)
    state.last_move = None
    return state
\end{lstlisting}
