\hypertarget{hilfsfunktionen-othello_test_util.ipynb}{%
\section{Hilfsfunktionen
(othello\_test\_util.ipynb)}\label{hilfsfunktionen-othello_test_util.ipynb}}

\label{sec:util}

\begin{lstlisting}[language=Python]
%%HTML
<style>
.container { width:100% }
</style>
\end{lstlisting}

\begin{lstlisting}[language=Python]
%run othello_game.ipynb
%run othello_ai.ipynb
\end{lstlisting}

\begin{lstlisting}[language=Python]
import time
import ipywidgets
\end{lstlisting}

Der folgende Code beinhaltet Hilfsfunktionen zum Testen und sammeln von
Statistiken.

Die Funktion \passthrough{\lstinline!debug\_num\_visited\_states!}
berechnet für einen Spielzustand den nächsten Zug mit verschiedenen
Algorithmen und misst dabei die benötigte Zeit. Zusätzlich werden mit
den globalen Variablen \passthrough{\lstinline!debug\_mm\_count!} und
\passthrough{\lstinline!debug\_ab\_count!} die Anzahl der überprüften
Zustände gezählt. Diese Zahlen geben einen Überblick darüber, wie viele
Zweige durch den Algorithmus ausgeschlossen werden konnten und nicht
überprüft werden mussten.

\begin{lstlisting}[language=Python]
def debug_num_visited_states(state, depth):
    global transposition_table
    global debug_mm_count
    global debug_ab_count
    global debug_pc_count

    # calculate next move with each algorithm and measure time
    debug_mm_count= 0
    start = time.time()
    ai_make_move(minimax, state, depth, combined_heuristic)
    secs = time.time() - start
    print("Minimax takes", secs, "s and evaluates the heuristic",
          debug_mm_count, "times")
    debug_ab_count= 0
    transposition_table = {}
    start = time.time()
    ai_make_move(alphabeta, state, depth, combined_heuristic)
    secs = time.time() - start
    print("AlphaBeta takes", secs, "s and evaluates the heuristic",
          debug_ab_count, "times")
    debug_ab_count = 0
    transposition_table = {}
    start = time.time()
    ai_make_move_id(alphabeta, state, depth, combined_heuristic)
    secs = time.time() - start
    print("AlphaBeta + ID takes", secs,
          "s and evaluates the heuristic", debug_ab_count, "times")
    debug_pc_count = 0
    transposition_table = {}
    start = time.time()
    ai_make_move(probcut, state, depth, combined_heuristic)
    secs = time.time() - start
    print("ProbCut takes", secs, "s and evaluates the heuristic",
          debug_pc_count, "times")
    
    debug_pc_count = 0
    transposition_table = {}
    start = time.time()
    ai_make_move_id(probcut, state, depth, combined_heuristic)
    secs = time.time() - start
    print("ProbCut + ID takes", secs, "s and evaluates the heuristic",
          debug_pc_count, "times")
\end{lstlisting}

Die Funktion \passthrough{\lstinline!get\_statistics!} ist dazu da,
mehrere Spiele zu berechnen, in denen zwei \acp{KI} gegeneinander
spielen und Statistiken darüber zu sammeln.
\passthrough{\lstinline!num!} ist die Anzahl der Spiele, die
durchgeführt werden sollen. Die weiteren Parameter legen fest, welche
\acp{KI} und welche Heuristiken für die Spieler verwendet werden sollen.
Statistiken werden nach jedem Spiel aktualisiert.

Da das Spielfeld nicht gezeichnet werden soll, wird statt
\passthrough{\lstinline!next\_move!} die Funktion
\passthrough{\lstinline!next\_move\_blind!} verwendet.

\begin{lstlisting}[language=Python]
def next_move_blind(state, settings):
    make_move = settings[state.turn]['mode']
    depth = settings[state.turn]['depth']
    timelim = settings[state.turn]['timelimit']
    heuristic = settings[state.turn]['heuristic']
    ai = settings[state.turn]['algorithm']
    intval = timelim if make_move == ai_make_move_id_timelimited else depth
    state = make_move(ai, state, intval, heuristic)
    if not state.game_over:
        state = next_move_blind(state, settings)
    return state
\end{lstlisting}

\begin{lstlisting}[language=Python]
def play_game(settings):
    state = GameState()
    state = next_move_blind(state, settings)
    return count_disks(state, BLACK), count_disks(state, WHITE)
\end{lstlisting}

\begin{lstlisting}[language=Python]
def get_statistics(num, settings):
    status = ipywidgets.widgets.Label()
    display(status)
    result = []
    wins = [0, 0, 0]
    status.value = f'0 / {num} games played, b/d/w: {wins[0]}/{wins[1]}/{wins[2]}'
    try:
        for i in range(num):
            (b, w) = play_game(settings)
            result.append((b, w))
            if b > w:
                wins[0] += 1
            elif w == b:
                wins[1] += 1
            else:
                wins[2] += 1
            status.value = f'{i+1} / {num} games played, b/d/w: {wins[0]}/{wins[1]}/{wins[2]}'
    except KeyboardInterrupt:
        status.value = f'Interrupted: {i} / {num} games played, b/d/w: {wins[0]}/{wins[1]}/{wins[2]}'
    print(result)
    return wins
\end{lstlisting}
