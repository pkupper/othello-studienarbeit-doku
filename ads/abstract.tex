%!TEX root = ../dokumentation.tex

\pagestyle{empty}

\renewcommand{\abstractname}{\langabstract}

\begin{otherlanguage}{english}
\begin{abstract}
Othello is a board game for two players. Due to the games deterministic nature, the strength of a player is solely
determined by their decision making and strategy.

The goal of this research project is to develop an artificial intelligence for the game Othello, which aims to be as
strong as possible, ideally beating human players as well as some publicly available AI agents. This
artificial intelligence is to be implemented as a Jupyter Notebook in the Python programming language, and can be played
against by a human player using a graphical interface.

The Minimax algorithm with depth-limited search and a heuristic for evaluating game states serves as a base for the
implementation. Alpha-beta pruning and the probabilistic forward cuts of ProbCut further optimize the performance and hence the strength of this algorithm. In addition, different features for use in the heuristic
function are tested and weighted.

The result of this project is a highly configurable artificial intelligence, that consistently wins against publicly
available AIs such as cowthello and two agents developed as part of a predecessor project. The AI is comfortably
configurable and usable through the user interface implemented in the Jupyter Notebook.

\end{abstract}
\end{otherlanguage}


\renewcommand{\abstractname}{\langabstract}

\begin{abstract}
Othello ist ein Brettspiel, welches von zwei Spielern gespielt wird. Da das Spiel völlig deterministisch ist, wird die
Stärke eines Spielers einzig und allein von dessen Entscheidungen und somit dessen Strategie bestimmt.

Ziel dieser Arbeit ist es, eine möglichst starke Künstliche Intelligenz für das Spiel Othello zu entwickeln, die sowohl
gegen menschliche Spieler als auch gegen einige öffentlich verfügbare KI-Agenten gewinnt. Die Implementierung findet in Form
eines Jupyter Notebooks in der Programmiersprache Python statt und stellt eine Grafische Benutzeroberfläche zur Verfügung, über
die ein menschlicher Spieler gegen die KI antreten kann.

Grundlage für die Umsetzung stellt der Minimax-Algorithmus mit begrenzter Tiefensuche unter Anwendung einer
heuristischen Evaluationsfunktion dar. Dessen Performance und somit Stärke werden durch die Verwendung von
Alpha-Beta-Pruning und ProbCut, einer statistischen Abschätzung, weiter verbessert. Außerdem beschäftigt sich die
Arbeit mit der Ermittlung und Gewichtung geeigneter Merkmale zur Berechnung der Heuristik.

Das Resultat dieser Arbeit ist eine stark konfigurierbare KI, welche gegen mehrere öffentlich
verfügbare Künstliche Intelligenzen für Othello, wie Cowthello und zwei in einer Vorgängerarbeit erarbeitete KI-Agenten, durchgängig gewinnt. Über die
Benutzeroberfläche im Jupyter Notebook lässt sie sich komfortabel konfigurieren und bedienen.
\end{abstract}
