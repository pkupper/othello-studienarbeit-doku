% !TeX root = ../dokumentation.tex

\addchap{\langanhang}

\section*{A. Gewichtung der aktuellen und potenziellen Mobilität}
 
\setcounter{table}{0}
\renewcommand{\thetable}{A\arabic{table}}

\begin{table}[hb]
\centering
\begin{tabular}{c|c|ccc}
\hline
Heuristik S\,/\,W & \#Spiele & Siege S & Unentschieden & Siege W \\
\hline
 A\,/\,B & 20 & 50\,\% &  2\,\% & 48\,\% \\
 A\,/\,C & 20 & 46\,\% &  4\,\% & 50\,\% \\
 A\,/\,D & 20 & 32\,\% &  8\,\% & 60\,\% \\
 B\,/\,C & 20 & 48\,\% &  2\,\% & 50\,\% \\
 B\,/\,D & 20 & 42\,\% &  2\,\% & 56\,\% \\
 C\,/\,D & 20 & 40\,\% &  2\,\% & 58\,\% \\
\hline
\end{tabular}
\caption{Einfluss aktueller und potenzieller Mobilität auf den Anteil der Siege}
\label{table:mobility}
\end{table}

\small{
A: Nur aktuelle Mobilität \\
B: Nur potenzielle Mobilität \\
C: Wechsel von potenzieller zu aktueller Mobilität bei 36 Steinen auf dem Spielfeld \\
D: Lineare Kombination von potenzieller und aktueller Mobilität}

\pagebreak

\section*{B. Anzahl der Steine als Heuristik am Ende des Spiels}
 
\setcounter{table}{0}
\renewcommand{\thetable}{B\arabic{table}}

\begin{table}[hb]
\centering
\begin{tabular}{c|c|ccc}
\hline
Verwendete Heuristik S\,/\,W & \#Spiele & Siege S & Unentschieden & Siege W \\
\hline
 A\,/\,B & 20 &100\,\% &  0\,\% &  0\,\% \\
 A\,/\,C & 20 & 90\,\% & 10\,\% &  0\,\% \\
 A\,/\,F & 20 & 20\,\% &  0\,\% & 80\,\% \\
 F\,/\,D & 20 & 60\,\% &  0\,\% & 40\,\% \\
 F\,/\,E & 20 & 75\,\% &  0\,\% & 25\,\% \\
 F\,/\,G & 20 & 90\,\% &  0\,\% & 10\,\% \\
\hline
\end{tabular}
\caption{Einfluss der Anzahl der Steine auf den Anteil der Siege}
\label{table:disccount}
\end{table}

\small{
A: Heuristik ohne Anzahl der Steine (Mobilität und Cowtello) \\
B: Anzahl der Steine ab 50 belegten Feldern \\
C: Anzahl der Steine ab 58 belegten Feldern \\
D: Anzahl der Steine ab 60 belegten Feldern \\
E: Anzahl der Steine ab 61 belegten Feldern \\
F: Anzahl der Steine ab 62 belegten Feldern \\
G: Anzahl der Steine ab 63 belegten Feldern}

\pagebreak
%\includepdf[pages=-,scale=.9,pagecommand={}]{Aufgabenstellung.pdf} % PDF um 10% verkleinert einbinden --> Kopf- und Fußzeile  werden so korrekt dargestellt. Die Option `pages' ermöglicht es, eine bestimmte Sequenz von Seiten (z.B. 2-10 oder `-' für alle Seiten) auszuwählen.

