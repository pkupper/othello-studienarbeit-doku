% !TeX root = ../dokumentation.tex

\addchap{\langanhang}

\section*{A. Gewichtung der aktuellen und potenziellen Mobilität}
 
\setcounter{table}{0}
\renewcommand{\thetable}{A\arabic{table}}

\begin{table}[hb]
\centering
\begin{tabular}{c|cccc}
\hline
Gewichtung der Mobilität S\,/\,W & Siege S & Unentschieden & Siege W \\
\hline
 A\,/\,B & 50\,\% &  2\,\% & 48\,\% \\
 A\,/\,C & 46\,\% &  4\,\% & 50\,\% \\
 A\,/\,D & 32\,\% &  8\,\% & 60\,\% \\
 B\,/\,C & 48\,\% &  2\,\% & 50\,\% \\
 B\,/\,D & 42\,\% &  2\,\% & 56\,\% \\
 C\,/\,D & 40\,\% &  2\,\% & 58\,\% \\
\hline
\end{tabular}
\caption{Einfluss aktueller und potenzieller Mobilität auf den Anteil der Siege}
\label{table:mobility}
\end{table}

\small{
A: Nur aktuelle Mobilität \\
B: Nur potenzielle Mobilität \\
C: Wechsel von potenzieller zu aktueller Mobilität bei 36 Steinen auf dem Spielfeld \\
D: Lineare Kombination von potenzieller und aktueller Mobilität}

\pagebreak
%\includepdf[pages=-,scale=.9,pagecommand={}]{Aufgabenstellung.pdf} % PDF um 10% verkleinert einbinden --> Kopf- und Fußzeile  werden so korrekt dargestellt. Die Option `pages' ermöglicht es, eine bestimmte Sequenz von Seiten (z.B. 2-10 oder `-' für alle Seiten) auszuwählen.

